\documentclass{article}
\usepackage[utf8]{inputenc}
%\usepackage{standalone}
% \usepackage[breaklinks=true,colorlinks=true,linkcolor=blue,citecolor=black,urlcolor=blue]{hyperref} % url
\usepackage[top=0.85in,left=1in,footskip=0.75in]{geometry}
\usepackage[dvipsnames]{xcolor} % colors
% \definecolor{blue-pigment}{rgb}{0.2, 0.2, 0.6}
\usepackage[colorinlistoftodos,textwidth=4.3cm]{todonotes} % for comments
\usepackage[normalem]{ulem} % for sout (strikeout)
\usepackage{url}

\usepackage[colorinlistoftodos]{todonotes} % for comments

\usepackage{lineno}
%\usepackage{xr-hyper} 
\usepackage{xr} 
%\usepackage{hyperref} 
\newenvironment{Reply}{\noindent\color{BlueViolet}\textbf{Reply.}}{\vspace{1em}}

%\usepackage{xr} % for getting lines with modifs from the other doc
%\externaldocument[ms-]{plos_latex_template}
% In your preamble

\makeatletter
\newcommand*{\addFileDependency}[1]{% argument=file name and extension
  \typeout{(#1)}
  \@addtofilelist{#1}
  \IfFileExists{#1}{}{\typeout{No file #1.}}
}
\makeatother

\newcommand*{\myexternaldocument}[1]{%
    \externaldocument{#1}%
    \addFileDependency{#1.tex}%
    \addFileDependency{#1.aux}%
}
\myexternaldocument{plos_latex_template}
%%%% defining colors for coauthors %%%%

\setlength{\parskip}{0.3cm plus 4mm minus 3mm} % space between paragraphs

\title{Ten simple rules to host an inclusive conference\\ \vspace{0.5cm}
\textbf{Response to reviewers}}
\author{}
\date{}


\begin{document}

\maketitle

When you are ready to resubmit, please upload the following:


[1] A letter containing a detailed list of your responses to all review comments, and a description of the changes you have made in the manuscript. Please note while forming your response, if your article is accepted, you may have the opportunity to make the peer review history publicly available. The record will include editor decision letters (with reviews) and your responses to reviewer comments. If eligible, we will contact you to opt in or out

[2] Two versions of the revised manuscript: one with either highlights or tracked changes denoting where the text has been changed; the other a clean version (uploaded as the manuscript file).

Important additional instructions are given below your reviewer comments.

 Thank you again for your submission to our journal. We hope that our editorial process has been constructive so far, and we welcome your feedback at any time. Please don't hesitate to contact us if you have any questions or comments.



Reviewer's Responses to Questions

Comments to the Authors:

\textbf{Reviewer \#1:}
I greatly enjoyed reading the paper "Ten simple rules to host an inclusive conference", and think it provides an easy to follow set of rules that organizers can use to ensure an inclusive conference. I found many tips for inclusiveness in the article to be especially helpful, such as multilingual accommodations and ensuring screen-reader accessibility.

I just have a few aspects I wasn't 100\% clear on.

(1) In Rule 4, the manuscript states "Define measurable indicators for these goals e.g. a gender distribution of your speakers that is representative of the general population, the participation of people from diverse races and ethnicities among the organizers, speakers, and attendees, or participation from key geographic regions", but how do organizers choose the correct demographics to monitor and the expected background rates? Should it match global, the local, or field-specific rates? Or should it be remediative, where it specifically selects for communities historically ignored in the field beyond the regional population rate?

\begin{Reply}
   We are going to refer to line numbers in the following way: In the plos\_latex\_template.tex document, add a line number label where you want to edit like this: \\ linelabel\{rev1-com1-a\}
   To refer to these lines in this document, add a reference here, like this: Lines   \ref{rev1-com1-a} to \ref{rev1-com1-b}
   
   %ast: We do not have a correct answer for this, unfortunately, and we believe there is a reflexion work that every team has to do in their context. We had extensive discussions about the indicators and our conclusion is that the "correct" demographic to monitor will vary for every conference, field, and local and regional context. 
   % Depending on the field, greater achievements can be made, and in some areas, this exercise will be more difficult and yet extremely important even if the apparent advancements are less evident. 
   % Our team is inclined to propose what the reviewer calls "remediative measures", or affirnative actions, and be explicitly more than inclusive, but we understand that not every team can set such targets. And even with bold intentions, there is no guarantee that the final result is going to be as expected. 
   % A support from the community may be a great way to strenghen bold goals. However, in some communities the common sense shared by members may represent a wish to preserve the status quo, so not always a broad support from the community can be expected -but this may indicate also a need to broaden the definition of the community.
   % We would like to encourage the readers of any field and regional context to make this exercise independently of the status of the field and the initial impression that these efforts won't be successful or sufficient. 
   % to edit: We have made this explicit in the text. We also included the fact that the community 1. should be defined and 2. can be consulted before the conference to make sure that this construction is collective and participative. 

%Marce: I agree with reviewer 1 that this should be remediative, quoting: "where it specifically selects for communities historically ignored in the field beyond the regional population rate". To respond to the same reviewer, I think we can mention how  UseR presented the demographics of the attendees/participants, and how this was shared with the community as a way to be held accountable.
%Marce: About rule 10: both reviewers 1 and 3 suggested to discuss further how to implement these rules. I think rule 10 is already discussing that in a clear manner.
\end{Reply}

(2) Adding on to the first point, how are the indicators measured? Should the conference organizers quantify the demographics of the attendees/participants? If so, should these details be published to the participants after the conference? How do organizers hold themselves accountable? And to whom are they held accountable to; the organizing committee themselves or the greater community?

\begin{Reply}
   % Yani: I think we can mention and link to the info board and to the blog post post conference analyzing this numbers like the Latin American one and the accessibility one. The surveys are also another source of data for this accountability. I think this is related with the other review asking about governance. 
   %ast: The tenth rule is directly connected to the first, because these indicators can be measured and evaluated via surveys, direct impressions from the atendance, and exercises such as focus groups. Of all these options, surveys are the most frequently used but we wanted to highlight that there are other options. This link to rule 10 was not explicit in the text (edit!)
   % About publishing the details after the conference, the privacy and anonymity of the participants should be respected, so some kind of binning/grouping of the data and care with categories that have a low number of respondents should be taken. %make sure this in the manuscript! we had this in previous versions.
   % Accountability of the team should be posible with or without public indicators. Open channels for communication and open fields and the possibility to make anonymous comments should be provided. Especially if there is a claim for diversity and inclusion, the general community should be able to give feedback about the success of these efforts. 
   %to edit: cite the tenth rule in the first, telling there will be accountability. but also say that the indicators should not become a goal themselves. 

\end{Reply}
(3) In Rule 4, it states "Check the final lists of speakers and do not hesitate to curate them to correct any imbalances you observe;" Does this pertain to invited speakers only, or speakers selected from submitted abstracts? If this pertains to scored abstracts, how does one ensure that this is done in a transparent and reproducible way? Or maybe I'm not fully grasping what curate means in this context?

\begin{Reply}
   %ast: Although curation is easier and definitely encouraged for invited speakers, we would like to reinforce that even transparent scored processes can be influenced by implicit bias, once the reviewers start evaluating there might be "shifting criteria" and other considerations entering mid-process.
   %We have added some extra explanation about this curation, that include at least: 
   %1. determining clear criteria before seeing the list of submissions, 2. examining the final list of selected work for imbalances in gender, geographical representation and other visible aspects of diversity. 3. check if close competitors (similar quality work) were excluded for some apparent bias.
   % This process is hard because there are many aspects of diversity that will be unreachable for reviewers and are prone to perceived cues rather than real information (so not every dicersity dimension can be considered here). The invitation is to examine similar quality work and try to understand why some work was included rather than other and to consider whether bias could have influenced the outcome.
   %criteria should be clear before the abstract evaluation. Some adaptation of this can be made: https://honors.agu.org/files/2014/11/AWARDS-SUGGESTED-BEST-PRACTICES-MAY2011.pdf
\end{Reply}

(4) In Rule 5, it states "The physical venue would have to include multiple ways to connect with online folks such as cameras and microphones to allow them to follow who is speaking at the in-person space, and tablets to help in-person attendees interact with remote participants." How do the tablets help in-person attendees interact with remote participants? Especially participants that may be in a different time-zone. Or are the tablets used to help the online activities mentioned in the next paragraph?

\begin{Reply}
   %ast: Although one can assume that an ever growing proportion of people have their own handheld device to interact online with the remote participants, we cannot assume that everyone does, or that the models are compatible with the options of the conference. Some devices (and support for those and the participants') should be provided by the organization. With everyone having a device, the idea is to unify the networking activities and Q & A sessions to be online, this more integrated with the remote component.
\end{Reply}

(5) In Rule 7 it states " Allow for abstract submission in both English and the language the person feels more comfortable with, and consider the possibility of assigning a reviewer who is fluent in that language." It is not trivial to submit an abstract in two languages, and if there is no chance that the non-English abstract won't be reviewed I feel like this would be wasting their time. Is it possible to denote the accepted languages ahead of time? Or for the submitter to put in a language request before the abstract deadline?

\begin{Reply}
%ast: "and" should be "or". Yes, definitely, rather than two abstracts, one in English and the other in a second language, the best way would be to be able to have abstracts in languages different than English without the need for a duplication. That depends on the conference capabilities but we would recommend both submission and presentation be open to other languages. The idea for a previous request/screening about the languages that will be needed is good. Another option is to have a language editing help for English, via the communities or via the organizing team, that helps ensure that the submitted abstracts are conveying the meaning the authors are trying to give. This could include also, abstracts written in English by students and early career researchers. 
\end{Reply}

Overall, I think these rules are great. My biggest concern is on the concept of organizing committee accountability. They are defining what the diversity goals are and how they are measured, but it isn't clear to me what input the community has on the conference. I feel like the greater community should be able to have input into the organizing process as well as the ability to see if their concerns are being addressed.

\begin{Reply}
%ast: We thank the reviewer's feedback. We have added some pre-conference participation of the community, also inspired by examples we got from colleagues at pycon Germany. The accountability is very hard to define because no formal structure requires that in the present, it has to be the team's will to be held accountable to their participant community. Clarity in the communication strategy and open spaces for feedback are key here.
\end{Reply}

\textbf{Reviewer \#2:}
L78-L80 As a female scientist I would ask myself "where are we" (because we still have parts of the World where female scientist are rare on leading positions), but I consider science apolitical in every segment, and I certainly suggest that it concentrate here on the scientific youth, the segment of research in industry, etc., and not on the above.
The rest of the paper it should certainly be praised because it enters into the essence of the exchange of knowledge today.

\begin{Reply}

\end{Reply}

\textbf{Reviewer \#3:} 
This is a timely, well-written commentary on organising an inclusive conference and should act as a reference compendium for researchers, practitioners as well as conference organisers.

I have some comments for authors to consider, and possibly expand upon.

1. As much as I enjoyed reading the ten rules, I missed finding mention of how this is to be implemented and governed. Therefore, it is poignant that authors should brief discuss having an inclusion oversight or governance committee that tracks these metrics and their implementation.

\begin{Reply}
%ast: my first hunch in general: the metrics part is one of the hardest part to implement and reviewer1 talked about it, but it is also one of the "goals" that stick with some readers and we were trying to avoid putting them at the center of decision-making because they can become a end in itself and not a way to understand if your efforts were successful
%The tenth rule should have accounted for this but we may need to be more specific in the messages of not letting yourself be prey of the indicators. 
%Team accountability and governance were part of rule 10 so maybe we could strengthen this.
\end{Reply}

2. Secondly, I also missed detailed methodology on how these ten rules were developed or consensus were reached at? Were stakeholders from diverse backgrounds consulted? Was the consultation process iterative? Were the recommendations thus made by those from diverse backgrounds incorporated? How were the final draft reached at - through consensus or voting or a process to that effect? This is important as for such a set of indications to be indicative, it is important that opinions from all backgrounds are taken into account and co-developed.

\begin{Reply}
   %Yani: I think this is not on the scope of the paper, but we can give an explanation of the process, which I think was fantastic and inclusive and have a diverse background on the construction of the rules. 
   %ast: i agree, this may break the format for a ten simple rules piece so maybe a very short mention in the introduction or as SupMat?
\end{Reply}

3. In Rule 1, though gender or ethnic diversity has been highlighted, geographical diversity needs to be discussed in detail. There is significant "north-south" divide in scientific research - with several of institutions as well as researchers from LMICs or global south not being given appropriate representation on conference committees as well as on speakers diversity.


4. In Rule 2, safe and inclusive environments, I find little attention to "racist" aberrations or incidents - and how they should be mitigated and when reported be appropriately addressed with fairness and justice.
Several speakers from ethnically diverse backgrounds face racism, exclusion or apathy at such conferences. This need to be discussed.

\begin{Reply}

%ast: All specific examples of offenses should be included in the code of conduct construction process, and that's why we didn't make a list here, but maybe we can make explicit mentions of what type of offenses can/have to be in the code of conduct. %suggestion for edits: add this, talk a little bit more about the content of the CoC text.


\end{Reply}

5. In Rule 4, via a vis bias, bias against people with disability and ethnic minorities should also be highlighted.

\begin{Reply}
%ast: We have tried to make a distinction between visible and invisible aspects of human diversity, and we have to respect the right of every individual to occupy the space and assert/disclaim who they are and the right to remain silent about some aspects of their lives. Rule 2 is about being proactive regarding all dimensions of diversity, and Rule 4 focuses more on the visible aspects of diversity, especifically those that can be grasped via written communication.
% We mention this because counteracting bias actions during the selection of speakers and submitted work, we--unfortunately--only have *perceived* cues of these dimensions. We can cite Laura's piece on video abstracts being exclusionary and be more explicit about the fact that perceived cues and implicit and explicit racism go hand in hand.
\end{Reply}

6. In Rule 8, inclusive communication strategy - it is important that the speakers from diverse backgrounds, especially from ethnic minorities etc are highlighted or given prominence in social media strategy or communication. Often, these speakers face relatively less focus. This may be discussed.

\begin{Reply}
   %Yani: I think all keynote deserve the _same_ communication, if we give prominence to minoritized group could be perceived as tokenism. So I would take the suggestion on NOT give more prominence to one group of keynote and make sure that all of them are promote in the same spaces and in the same way.  We can mention here that communities partners can help with the highlight of the minoritazed groups (tricky).
   %ast: I agree with Yani, and we had it partially covered in the rule 7, "Promote these sessions among all conference attendees and announce if captions will be available; non-English sessions should be given the same relevance as the rest of the program." but we can add something here. 
\end{Reply}

7. In Conclusion, suggest also adding that issues of diversity and inclusion have come to fore even more during COVID-19 as we witness exacerbation of inequities disadvantaging those from vulnerable backgrounds (see https://pubmed.ncbi.nlm.nih.gov/33343410/
PMID: 33343410)


\end{document}

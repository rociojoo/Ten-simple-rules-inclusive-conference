% Template for PLoS
% Version 3.5 March 2018
%
% % % % % % % % % % % % % % % % % % % % % %
%
% -- IMPORTANT NOTE
%
% This template contains comments intended 
% to minimize problems and delays during our production 
% process. Please follow the template instructions
% whenever possible.
%
% % % % % % % % % % % % % % % % % % % % % % % 
%
% Once your paper is accepted for publication, 
% PLEASE REMOVE ALL TRACKED CHANGES in this file 
% and leave only the final text of your manuscript. 
% PLOS recommends the use of latexdiff to track changes during review, as this will help to maintain a clean tex file.
% Visit https://www.ctan.org/pkg/latexdiff?lang=en for info or contact us at latex@plos.org.
%
%
% There are no restrictions on package use within the LaTeX files except that 
% no packages listed in the template may be deleted.
%
% Please do not include colors or graphics in the text.
%
% The manuscript LaTeX source should be contained within a single file (do not use \input, \externaldocument, or similar commands).
%
% % % % % % % % % % % % % % % % % % % % % % %
%
% -- FIGURES AND TABLES
%
% Please include tables/figure captions directly after the paragraph where they are first cited in the text.
%
% DO NOT INCLUDE GRAPHICS IN YOUR MANUSCRIPT
% - Figures should be uploaded separately from your manuscript file. 
% - Figures generated using LaTeX should be extracted and removed from the PDF before submission. 
% - Figures containing multiple panels/subfigures must be combined into one image file before submission.
% For figure citations, please use "Fig" instead of "Figure".
% See http://journals.plos.org/plosone/s/figures for PLOS figure guidelines.
%
% Tables should be cell-based and may not contain:
% - spacing/line breaks within cells to alter layout or alignment
% - do not nest tabular environments (no tabular environments within tabular environments)
% - no graphics or colored text (cell background color/shading OK)
% See http://journals.plos.org/plosone/s/tables for table guidelines.
%
% For tables that exceed the width of the text column, use the adjustwidth environment as illustrated in the example table in text below.
%
% % % % % % % % % % % % % % % % % % % % % % % %
%
% -- EQUATIONS, MATH SYMBOLS, SUBSCRIPTS, AND SUPERSCRIPTS
%
% IMPORTANT
% Below are a few tips to help format your equations and other special characters according to our specifications. For more tips to help reduce the possibility of formatting errors during conversion, please see our LaTeX guidelines at http://journals.plos.org/plosone/s/latex
%
% For inline equations, please be sure to include all portions of an equation in the math environment.  For example, x$^2$ is incorrect; this should be formatted as $x^2$ (or $\mathrm{x}^2$ if the romanized font is desired).
%
% Do not include text that is not math in the math environment. For example, CO2 should be written as CO\textsubscript{2} instead of CO$_2$.
%
% Please add line breaks to long display equations when possible in order to fit size of the column. 
%
% For inline equations, please do not include punctuation (commas, etc) within the math environment unless this is part of the equation.
%
% When adding superscript or subscripts outside of brackets/braces, please group using {}.  For example, change "[U(D,E,\gamma)]^2" to "{[U(D,E,\gamma)]}^2". 
%
% Do not use \cal for caligraphic font.  Instead, use \mathcal{}
%
% % % % % % % % % % % % % % % % % % % % % % % % 
%
% Please contact latex@plos.org with any questions.
%
% % % % % % % % % % % % % % % % % % % % % % % %

\documentclass[10pt,letterpaper]{article}
\usepackage[top=0.85in,left=2.75in,footskip=0.75in]{geometry}

% amsmath and amssymb packages, useful for mathematical formulas and symbols
\usepackage{amsmath,amssymb}

% Use adjustwidth environment to exceed column width (see example table in text)
\usepackage{changepage}

% Use Unicode characters when possible
\usepackage[utf8x]{inputenc}

% textcomp package and marvosym package for additional characters
\usepackage{textcomp,marvosym}

% cite package, to clean up citations in the main text. Do not remove.
\usepackage{cite}

% Use nameref to cite supporting information files (see Supporting Information section for more info)
\usepackage{nameref,hyperref}

% line numbers
\usepackage[right]{lineno}

% ligatures disabled
\usepackage{microtype}
\DisableLigatures[f]{encoding = *, family = * }

% color can be used to apply background shading to table cells only
\usepackage[table]{xcolor}

% array package and thick rules for tables
\usepackage{array}

%strikethrough
\usepackage{soul}

% create "+" rule type for thick vertical lines
\newcolumntype{+}{!{\vrule width 2pt}}

% create \thickcline for thick horizontal lines of variable length
\newlength\savedwidth
\newcommand\thickcline[1]{%
  \noalign{\global\savedwidth\arrayrulewidth\global\arrayrulewidth 2pt}%
  \cline{#1}%
  \noalign{\vskip\arrayrulewidth}%
  \noalign{\global\arrayrulewidth\savedwidth}%
}

% \thickhline command for thick horizontal lines that span the table
\newcommand\thickhline{\noalign{\global\savedwidth\arrayrulewidth\global\arrayrulewidth 2pt}%
\hline
\noalign{\global\arrayrulewidth\savedwidth}}


% Remove comment for double spacing
%\usepackage{setspace} 
%\doublespacing

% Text layout
\raggedright
\setlength{\parindent}{0.5cm}
\textwidth 5.25in 
\textheight 8.75in

% Bold the 'Figure #' in the caption and separate it from the title/caption with a period
% Captions will be left justified
\usepackage[aboveskip=1pt,labelfont=bf,labelsep=period,justification=raggedright,singlelinecheck=off]{caption}
\renewcommand{\figurename}{Fig}

% Use the PLoS provided BiBTeX style
\bibliographystyle{plos2015}

% Remove brackets from numbering in List of References
\makeatletter
\renewcommand{\@biblabel}[1]{\quad#1.}
\makeatother



% Header and Footer with logo
\usepackage{lastpage,fancyhdr,graphicx}
\usepackage{epstopdf}
%\pagestyle{myheadings}
\pagestyle{fancy}
\fancyhf{}
%\setlength{\headheight}{27.023pt}
%\lhead{\includegraphics[width=2.0in]{PLOS-submission.eps}}
\rfoot{\thepage/\pageref{LastPage}}
\renewcommand{\headrulewidth}{0pt}
\renewcommand{\footrule}{\hrule height 2pt \vspace{2mm}}
\fancyheadoffset[L]{2.25in}
\fancyfootoffset[L]{2.25in}
\lfoot{\today}

%% Include all macros below

\newcommand{\lorem}{{\bf LOREM}}
\newcommand{\ipsum}{{\bf IPSUM}}

%% END MACROS SECTION


\begin{document}
\vspace*{0.2in}

% Title must be 250 characters or less.
\begin{flushleft}
{\Large
\textbf\newline{Ten simple rules towards an inclusive conference} % Please use "sentence case" for title and headings (capitalize only the first word in a title (or heading), the first word in a subtitle (or subheading), and any proper nouns).
}
\newline
% Insert author names, affiliations and corresponding author email (do not include titles, positions, or degrees).
\\
Name1 Surname\textsuperscript{1,2\Yinyang},
Name2 Surname\textsuperscript{2\Yinyang},
Name3 Surname\textsuperscript{2,3\textcurrency},
Name4 Surname\textsuperscript{2},
Name5 Surname\textsuperscript{2\ddag},
Name6 Surname\textsuperscript{2\ddag},
Name7 Surname\textsuperscript{1,2,3*},
with the Lorem Ipsum Consortium\textsuperscript{\textpilcrow}
\\
\bigskip
\textbf{1} Affiliation Dept/Program/Center, Institution Name, City, State, Country
\\
\textbf{2} Affiliation Dept/Program/Center, Institution Name, City, State, Country
\\
\textbf{3} Affiliation Dept/Program/Center, Institution Name, City, State, Country
\\
\bigskip

% Insert additional author notes using the symbols described below. Insert symbol callouts after author names as necessary.
% 
% Remove or comment out the author notes below if they aren't used.
%
% Primary Equal Contribution Note
\Yinyang These authors contributed equally to this work.

% Additional Equal Contribution Note
% Also use this double-dagger symbol for special authorship notes, such as senior authorship.
\ddag These authors also contributed equally to this work.

% Current address notes
\textcurrency Current Address: Dept/Program/Center, Institution Name, City, State, Country % change symbol to "\textcurrency a" if more than one current address note
% \textcurrency b Insert second current address 
% \textcurrency c Insert third current address

% Deceased author note
\dag Deceased

% Group/Consortium Author Note
\textpilcrow Membership list can be found in the Acknowledgments section.

% Use the asterisk to denote corresponding authorship and provide email address in note below.
* correspondingauthor@institute.edu

\end{flushleft}
% Please keep the abstract below 300 words
\section*{Abstract (optional from what I've seen)}

The authors of this article participated in the organization team of the annual user conference of the R Project for Statistical Computing, held in July 2021. useR! conferences are non-profit events organized by volunteers from the R community and arranged by the R Foundation. The conference attracts a broad range of participants from academia, industry, government, and the non-profit sector. For 2021, we aimed to build a high-quality virtual and explicitly global conference in a kind, inclusive, accessible, and welcoming environment for everyone. 
In this article, we streamline our most important learnings in 10 simple rules to host an inclusive conference. These rules apply equally to academic, industry, or mixed conferences; the rules are inspired by a global experience but also apply at the regional or local level.

%andrea: matt's original sentence: reaching users and developers of the R language from more than 120 countries. 



% % Please keep the Author Summary between 150 and 200 words

\linenumbers

\section*{Introduction}

Conferences are spaces to meet and reconnect with members from a specific community, learn about advances in the field, and share our recent contributions. 
However, conferences are likely to reproduce the systematic discrimination occurring in other spaces in our fields.
An exclusionary conference experience can divert career paths, affect lives, and drive people out of academia \cite{biggsAcademicConferenceChilly2018}.
Efforts to address systematic discrimination in conferences and make them inclusive spaces can have a meaningful impact on our communities and the lives and careers of their members. 

This article suggests rules to pivot traditional conferences towards diversity and inclusion, and strive to build more inclusive and welcoming communities. 
Some scholarly or technological communities are formally structured as learned societies (e.g. the International Geographical Union or the Royal Statistical Society), others are composed by networks of less formal local groups (e.g. R User Groups or Python User Groups), and others may not have any established structure. 
For the latter, conferences may play even a bigger role in shaping the community, since there is no other organization or institution to gather the community members. 

The rules written here are directed to people who are part of a stable meetings committee that oversees the site/location selection process, or that coordinates with the local organizers of conferences.
The rules can also be helpful to local/virtual organizers who desire to make an inclusive conference starting at the planning stage.
These tips stem from the authors' experience of organizing useR! 2021, a virtual and global statistical computing conference for users and developers of the R programming language \cite{r_core_team_2021}. 
We embraced the challenge of organizing a high-quality virtual conference in the context of the COVID-19 pandemic and making it a kind, inclusive, and accessible experience for everyone. 
Here, we share the lessons learned within the past year of organizing this global useR! 2021, summarized as ten simple rules towards an inclusive conference.
We have organized these rules in three groups.
Rules 1, 2 and 10 refer to pillars of an inclusive conference: embracing diversity in all its dimensions, creating a safe and welcoming environment for everyone, and making the conference part of a long-term process for inclusion.
The next two rules are focused on the people that participate in the conference: 
Rule 3 refers to the importance of working with an inclusive and diverse organizing team, and Rule 4 concerns the necessity of removing implicit and systemic bias from spotlight roles like keynote speakers, other presenters, program committee members, or other session chairs. 
Rules 5 to 9 are rules about actions regarding components of the conference that should be carefully planned for: an online component, accessibility to people with disabilities, language inclusiveness, a welcoming communication strategy, and financial resources to support inclusion. 
These rules apply equally to academic, industry, or mixed conferences; the rules are inspired by a global experience but also apply at the regional or local level.
% Rocío: the last sentence was copied from the abstract because we need a closing sentence.
% Rocío: Could add a line introducing Fig. in https://docs.google.com/drawings/d/1iS1pLc9OldLMe_jhNT6OGIbT9s5KvlHpUoDd19vHkCY/edit?usp=sharing and the whole structure [x]

% keeping references:
%   However, conferences can become discriminating spaces, in which members of some specific privileged groups reproduce the systematic inequalities that occur in academia and society \cite{arendDisparityConferenceRegistration2019, timperleyHeMoanaPukepuke2020, gewinWhatScientistsShould2019, brownAbleismAcademiaWhere2018}.
% Liz adding reference marks2021meeting

\section{Embrace all dimensions of diversity}
\label{rule_diversity}

Diversity encompasses multiple dimensions: age, physical ability, career stage, gender, gender identity, geographic origin, language, neurodiversity, race, religion, sexual orientation, and socioeconomic background, to name a few. 
Human diversity should be celebrated and respected in every way. Nonetheless, we live in a world with implicit hierarchies along these axes. Some statuses (e.g., cisgender, white, male, from the US or Europe) hold the privilege of being the default settings for which all systems--including conferences--are consciously and unconsciously built.  
While no isolated initiative can change reality by itself, building a more diverse and inclusive conference starts by recognizing that these inequalities have systematically excluded whole groups of people from academia and scientific and professional circles \cite{timperleyHeMoanaPukepuke2020}. 

To be proactive allies in our conferences and communities, we should start by examining our own privileges (unearned advantages given by society to some people but not all).
Privilege is largely invisible to those who have it. And we will not be able to help end discrimination and oppression if we do not recognize our privileges first and the injustices that come with them.
Once we understand our own privileges, we can begin addressing them on an individual, collective, and institutional basis.
Recognizing our privileges--particularly in our field and in our scientific or professional community--will help identify which subgroups have been the most excluded or discriminated against. 
These are the groups we need to make more effort to include. 
Depending on the field, region, or community, it could be Black People, LGBTQIA+, Muslim, or others. 
%ast: ö check phrasing, add "people"
Intersectionality, or the complex, cumulative way in which the effects of multiple forms of discrimination combine or intersect, should be taken into account. 

Investing more effort in the most excluded groups does not mean neglecting the others (see \textbf{Rule \ref{rule_inclusion}} for more on that). 
But it will prevent from falling into the mistake of treating diversity as a checklist.
It will also guide the vision of diversity for your conference--and your strategies to achieve it:
Will a more diverse conference translate into an even gender distribution in your speakers? 
Would it be the presence of racialized people --especially Black people-- among the head organizers, speakers, and attendees? 
Would it be having LGBTQIA+ friendly-spaces or community participation from key geographic regions?
This exercise will also help coming up with indicators that would help the organizing team assess the fulfillment of the diversity objectives along the way. 

%batool: There are a few references which define privilege and might be good to add here including: Friedman, S., O’Brien, D., & McDonald, I. (2021). Deflecting Privilege: Class Identity and the Intergenerational Self. Sociology, 55(4), 716–733. https://doi.org/10.1177/0038038520982225
%batool: Another one is: Crevani, L. (2019). Privilege in place: How organisational practices contribute to meshing privilege in place. Scandinavian Journal of Management, 35(2). https://doi.org/10.1016/j.scaman.2018.09.002
%batool: please note that I didn't read both entire papers I mentioned above.
%sara: reference for privilege: McIntosh, Peggy. "White privilege: Unpacking the invisible knapsack." (1988).
% Rocío: I feel like we don't need to cite any reference about privilege the way it is now. But feel free to disagree.



\section{Create a safe and welcoming environment}
\label{rule_inclusion}

While it is essential to improve representation towards some of the most visible dimensions of human diversity, such as race, gender, and country of origin, building a truly inclusive environment means taking care of all the other aspects of diversity as well. Having consideration of religious practices, setting specific accommodations for breastfeeding and child care, having LGBTQIA+-friendly spaces, creating community-only spaces, enforcing the use of pronouns, and acknowledging that gender is not binary are just some examples of decisions that can make inclusion real.
Importantly, you can take active steps in creating a more welcoming environment without requiring anyone to disclose personal information.

Paying attention to some dimensions of diversity while neglecting others may have unintended, even harmful consequences. Lack of representation, unwelcoming--or overtly aggressive-- environments hinder participation (or future participation) of people who could otherwise become active community members. A negative conference experience can divert career paths, affect lives, and exclude people from some fields \cite{biggsAcademicConferenceChilly2018}. 
%ast maybe this last sentence could go in the introduction because it's general i marked [bad experience] in the intro

Adopting a code of conduct and creating a team to enforce it are key aspects in creating a safe environment during a conference \cite{favaroYourScienceConference2016}.
The code of conduct is a document meant to keep the community safe and should state clearly: the unacceptable behaviors, the spaces of the conference in which it applies, the consequences for engaging in unacceptable behavior, and the way to report violations \cite{auroraHowRespondCode2019}. 
The code of conduct should be displayed prominently in several spaces of the conference to deter people from unacceptable behavior.
An efficient code of conduct acts as a protection for the community because the people who are the target of unacceptable behavior tend to be the ones with less power or privilege.

The code of conduct team should receive training on how to receive reports, respond to incidents, and communicate their responses, and organize accordingly. A diverse code of conduct team will be more understanding of power dynamics and sensitive to discrimination and harassment issues. There are also disciplinary cultures and geographical considerations that may need to be considered when developing and communicating the code of conduct (CoC). We strongly recommend reading `How to Respond to Code of Conduct Reports' \cite{auroraHowRespondCode2019} as an excellent starting point for the Code of Conduct teamwork.



\section{Have an inclusive and diverse organizing team}
\label{rule_organizing_team}

A genuinely inclusive conference can only be organized by an inclusive and diverse organizing team. Build a team with people from different regions, genders, races, socioeconomic statuses, and other aspects of diversity. 
Go beyond balancing all genders in this effort, and pay attention to other marginalized groups (see \textbf{Rule \ref{rule_diversity}}). To ensure a deep understanding and smooth communication with different diverse groups, it is essential to create a representative working group that functions as a snapshot of the community at large. If you already have an organizing team, check for gaps in its composition. 

%ast: I added race -even if race is a construct- because focus is given to gender and the spaces tend to be predominantly white and US/Europe-centric. In other parts of the text I was using racialized but here we can keep it simple, i think. another thing: i avoided "gender balancing" because idk if that comes across as too binary, even if what is done is adding cis women to the team
% Rocío: what about ethnicity instead of race? I agree with the gender balancing thing.

% Rocío: Idea of this paragraph: build a diverse team

Gathering a diverse team will only work if there is real inclusion. People with disabilities often say: `Nothing about us without us'; the same holds for other dimensions of diversity. This means that the actual life experiences, expertise, and insights from people in marginalized groups cannot be replaced by good intentions from people outside these groups [cite design justice]. A truly inclusive and welcoming space is one in which everyone in the team is invited and allowed to bring their experience to bear. 

% Rocío: Idea of this paragraph: real inclusion when having diverse members

Creating and maintaining such a team and space may seem more challenging than working in homogeneous teams, but the positive outcomes are worthwhile. 
Having diverse people in decision-making positions will affect positively all the other aspects of your conference because all the processes will benefit from their input, expertise, and distinct perspectives \cite{hongGroupsDiverseProblem2004}. In addition, a diverse team plays an important role at creating a welcoming space because representation--seeing people with similar life experiences occupy public spaces, positions of power, and breaking negative stereotypes--is one of the best ways to create a sense of belonging (see \textbf{Rule \ref{rule_inclusion}}). 
% Rocío: Idea of this paragraph: positive outcomes of having a diverse team - keeping it general at least for now

%This is specially important for networking spaces, where receiving ideas, direct feedback, and spaces to co-lead from a wide variety of community members can be the key to organize meaningful sessions for all the conferences attendees. This gave a sense of variety and an appeal of "at least one networking session for everyone".  (Marcela)
%ast 17sep: i think we can talk here about planning neworking here, but also: the whole paragraph applies to team dynamics in general. Sara: maybe you can use this as we spoke relating team work here. And I think the working together part is not here but the following paragraph 
%sara: The last sentence in the paragraph sounds odd to me, but maybe I'm missing something. Andrea, I added a few sentences to the following paragraph regarding work dynamics
% Rocío: the rule is getting too long and, since this is focused on networking, I'd prefer it to go to a networking section.

% Liz: I know this section is getting long, and maybe this is too blunt, but: 
% People who have experience with exclusion have the social and technical 
% expertise about removing barriers to inclusion. Actively ask where the barriers are and how to mitigate them.
% Rocío: let's make sure this is included somewhere.

The following advice should apply to any kind of team, but is specially relevant when working with people from marginalized groups. First, rethink power relationships inside the group and share power. Be committed to examine power relations inside your organizing team and challenge it during all the process. % Rocío: Could we explain power relationship / share power ideas in more simple terms?
Don't expect self-nomination and voting to work as mechanisms to counteract systemic inequalities. Nominate directly and offer leading positions, and let people from privileged groups step down. 
Build an environment in which every person can express their position and give priority to people from systematically excluded groups.
% Liz: Use of "voice" in this way makes deaf people feel excluded. Maybe it could be something like "Build an environment in which every person has input..." Or anything as long as it doesn't suggest you have to have a voice to participate.
%ast: i rephrased slightly, avoiding "support" it's not a question of privileged people giving support but of positions being given importance
Offer support and guidance if you encounter cases of impostor syndrome, doubts about the use of English in communication (See \textbf{Rule \ref{rule_language}}), or others. You can, for example, break the expectations about leadership as a lonely task and create smaller, co-led groups, where everyone finds their preferred tasks and gets to take leadership. Splitting the workload and responsibilities should not be done by putting care-taking labors--community building, meeting organization and note-taking, conversations with potential partners--on the hands of women and other minoritized groups, while people from privileged groups take the lead in stereotypical highly-valued tasks (see \textbf{Rule \ref{rule_unbias}}). Make work visible, discuss collectively what should be done (and why), divide and account work. Being transparent and sharing the information is key to make work visible and collective. %sara: this relates to make labor visible from DF, maybe we can cite them here too. %Rocío: sure.
% Rocío: Idea of this paragraph: tips to make a diverse and inclusive team work.

% Liz: I have a little concern about mentoring WRT English, but I'm a white English speaker, so take it with a grain of salt. I think people who look like me need to do more accepting and less mentoring around English. If someone explicitly wants my help, I'm happy to give it because I love words and language, and I might learn something cool about another language. But I'm not sure we should be telling the English-priveleged readers that they are language mentors.
%ast. i agree with liz and wrote this from the point of view of a non-native speaker- in terms of support rather than mentorship. replaced with support but feel free to check
Most importantly, take care of your team. 
Having a diverse team and executing inclusive and accessible practices throughout the organizing period--it might be a year or two--may require a lot of effort. The effort is worth it because it strengthens the event and the community making it truly welcoming for everyone. 
However, having a strong community as the only reward may be enough for those who are in more privileged position. 
Some people and often the minoritized ones do not have institutional support to put time and effort into the organization tasks and do not have the luxury to commit to the organization for free. 
% Liz: Yes! This is especially true when *some* people are paid to work on the conference. I was just asked to do accessibility work that should have been done by the contracted, paid organizer. This is hugely problematic and will be true for lots of conferences outside the open software community.
In addition, tasks such as receiving and responding to code of conduct reports, can be emotionally intense work and should be additionally rewarded.
Prioritize your team's well-being. Check on them regularly and make sure everyone is comfortable. 
Be mindful of each particular context, be flexible with hours or commitments, and revise your budget (see \textbf{Rule \ref{rule_financial}}). 
% Rocío: Idea of this paragraph: take care of your team.
% Liz: also, include people from diverse backgrounds in all aspects of planning, not just the diversity-related ones, unless that is their only interest. In some organizations, I could be on the scientific committee and participate in many other ways, but I'm only called on for accessibility.


\section{Consciously unbias your spotlight roles}
\label{rule_unbias}

%idea1: our lists are biased

When choosing or inviting people as keynote speakers, program committee, session chairs, and other spotlight roles, it is likely that there will not be much diversity in the first set of names. Many of our biased lists are products of the existing systems that have always privileged some groups of people \cite{dwyerNoticeWhoScience2021,swartzScienceValueDiversity2019,wongBuildDiversityScience2020,dignazioUnicornsJanitorsNinjas2020}. Rather than deter us, this implicit and systemic bias should encourage us to look further to find great people that are not routinely in the spotlight. 
Ensuring diversity in each of these roles needs to be a deliberate process. We need to go beyond our narrow and often limited networks to look for, reach out to, invite, encourage, and onboard these people until there is ample representation across the diversity spectrum and dimensions. 

Make sure that every selection committee--the committee looking for keynote speakers, the selection committee for abstracts, the prizes and award committees--are also diverse \cite{swartzScienceValueDiversity2019, wongBuildDiversityScience2020}, and ask them to be aware that everyone has implicit biases, to recognize them, and try to counteract them. 
An inclusive and diverse organizing team (\textbf{Rule \ref{rule_organizing_team}}) is already a great starting point to overcome this bias in other roles. The regional and local communities in your field are also good sources to tap into. 
For example, for useR!2021, groups like AfricaR (africa-r.org), R-Ladies (rladies.org), MiR (mircommunity.com), Forwards (forwards.github.io), and LatinR (latin-r.com) were fundamental to reach people for the organizing team, potential presenters, to co-lead social events, and attendees; some of them even held spaces to help the members of their groups to prepare abstracts for submission.
When organizing an international academic conference, you may reach local associations, student organizations, or other Early Career Researchers groups or communities of practice: ask for their barriers for participation, and explicitly invite them to contribute to the conference. 
% Rocío: Not everyone knows what a community of practice is. Could we add a definition or use other words? Definition: "Communities of practice are groups of people who share a concern, a set of problems, or a passion about a topic, and who deepen their knowledge and expertise in this area by interacting on an ongoing basis". (Wenger, E., McDermott, R. A., & Snyder, W. (2002). Cultivating Communities of Practice: A Guide to Managing Knowledge. Harvard Business Press.)
Since many conferences are held with the primary goal of displaying cutting-edge work led by senior scientists and academics, early career researchers are often disadvantaged. Diversity should spread among all speaking roles, and the conference should also help raise the profiles of early career researchers and people from minoritized groups, fostering collaboration and building their skills.
% Liz: maybe consider a broader group than "early career researchers." One problem is that marginalized people have many fewer opportunities, and as they age that attenuation of opportunities compounds. Someone can be in the margins even years after finishing grad school or being a postdoc (not to mention that people without degrees could have contributions.


Furthermore, consider bringing to the spotlight the people that have contributed to your field in more collaborative ways \cite{cheng2020x+}. You can, for instance, reframe the awards ceremony to acknowledge those who prepared accessible slides and presentations (see Rule \ref{rule_accessibility}), for being mindful of inclusiveness. Community building, for instance, is challenging and usually unrewarded when compared to publications or software development \cite{acionWhyChooseCommunity2020}; give the people who deliberately took the time to work on their communities the recognition they deserve. Defy the stereotypical criteria for success by acknowledging these community practices. 
Finally, do not restrict people from marginalized groups or community-builders to talk or work only in issues related to Diversity, Inclusion, and Accessibility. Recognize their areas of expertise and respect their will regarding participation (or not) in community-building events.
% Rocío: Since the system usually rewards the most competitive people, appealing cvs and so on, I think it's worth it to explain that rewarding collaborative work is also part of unbiasing; not only because it's mostly done by women, people of color, etc., but because this kind of amazing work is usually neglected in our conferences. It may not be obvious to everyone that the content of this paragraph is not only "bring these people to the spotlight", but that it also means unbiasing the conference. I don't think I'll have time to add lines about this in the beginning of the paragraph so if anybody could do that, it would be great. 

% Liz: I might be able to, but I'm not sure where you want the additional text; where the beginning of the paragraph is because of how this is formatted. I couldn't agree more that being truly inclusive means not restricting people's role. Every subteam should be diverse, in addition to the whole organizing team. It's not just about representation but needs to encompass integration across the whole event. 

% ast: i agree a lot with Rocío here, the change in what is given importance is a key part of the unbiasing. i had the impression that the part of not restricting people's role goes above, in team rule. 

\section{Have a strong online component of the conference} 
\label{rule_online}

In-person interaction in conferences is priceless, but only for the ones who can afford to attend. 
Barriers such as cost of registration, transport and accommodation, the logistics of long-distance travel, and discriminatory visa applications, are particularly true for conferences that usually take place in high-income countries \cite{arendDisparityConferenceRegistration2019,gewinWhatScientistsShould2019}. 
% Rocío: we can add our Nature Correspondence letter to the citations later on. 
Online conferences are more inclusive: they do not require a visa or a big budget, and are more accessible to people who may be unable to travel because of health issues or family responsibilities \cite{salibaGettingGripsOnline2020}.
This means that online conferences have a greater reach, not only in terms of participants but in terms of the tutors and presenters that can participate \cite{atkinsonJournalMedicine20202021, roosOnlineConferencesNew2020}.
% Rocío: we can add our Nature Correspondence letter to the citations here too. I would've liked one that talks about increase in participation of tutors and presenters but haven't found any. 
The online format may also make it easier to be inclusive of geographic regions by encompassing several timezones in the week or easily deciding to rotate the favored timezone year to year without depending on a conference central location. 
% Rocío: main idea of the paragraph: online conferences are more inclusive
% Liz: but only if you make them that way. The R conference was a standout and innovative compared to others I've considred attending since Covid. 

Networking and socializing in online conferences may be challenging, mostly because we are used to personal interactions at coffee or lunch breaks, or other in-person social settings \cite{salibaGettingGripsOnline2020}. 
However, social events in in-person conferences can be exclusionary (e.g. galas or dinner nights at expensive venues), and they are a common place for code of conduct violations to occur \cite{auroraHowRespondCode2019}. Organizers should invest time in creating opportunities to meet and bond virtually. 
An online event is a great opportunity to create spaces for people who do not enjoy in-person networking. 
Some ideas in this line are: offering the option of written chat only, instead of voice or video conversations, opening events with teamwork like trivia, offer some events that can be enjoyed passively like movies, yoga sessions, or art displays, where attendants can choose to just sit and enjoy without talking, or have a chat channel to comment on their experiences during the session. Respect people's limits, preferences, and remember that 'the usual' does not necessarily work for everyone in your community, and that one single networking activity will never serve the whole community. It is worth trying new and varied activities that might work with subgroups of attendees.
Moreover, virtual communication may make people from minoritized groups feel more included, thus participate more (e.g. \cite{trianaDoesOrderFacetoFace2012}).
% Rocío: Networking paragraph.

In addition to being more inclusive, an online format is more environmentally-friendly since it eliminates travel-related emissions \cite{sarabipourChangingScientificMeetings2021,ninerBetterWhomLeveling2021, gattrellComparisonCarbonCosts2021}. 
Besides, if conference materials are systematically stored and pre-recorded, they can be made available to people who missed them live or even after the conference is over. 
% Rocío: Short paragraph of additional advantages of online conferences that are not about inclusion.

Alternatively, a conference could have a hybrid format with an in-person and an online component. This dual format could allow a group of people to interact face-to-face while providing many others the opportunity to participate remotely. The challenge and requirement for this kind of setting would be to make the online component as relevant as the in-person component and not just a consolation prize to the less privileged in the community \cite{ninerBetterWhomLeveling2021}.
Needless to say, an online component is not inclusive \textit{per se} without taking many other precautions, and it does not eliminate the need to organize an inclusive in-person component. 
%ast: I added Liz's suggestion and rephrased slightly to convey that online is not automatically inclusive
For instance, to avoid favoring the same regions in every edition, we suggest rotating the location of the conference—if it is global, moving from one continent to another. 
% Rocío: Hybrid conferences paragraph.

\section{Make the conference accessible to people with disabilities}
\label{rule_accessibility}

% Rocío: Could start with an introduction (or put somewhere else) that online can easily make conferences more inclusive for many people, but not necessarily for people with disabilities, and that they can be easily left behind unless we actively work on their inclusion. 
% Liz: let me know if you want me to take a shot at this.adding reference marks2021meeting

Conferences are among the least accessible spaces that people with disabilities may encounter \cite{priceAccessImaginedConstruction2009}. Even when the organizing team implements other inclusive practices, the concern about the participation of people with disabilities tends to be partial, and is one of the . 

Importantly, accessibility practices are inclusive not only for people with disabilities but can be beneficial to a broad spectrum of people. For instance, having captions is helpful to deaf and hard-of-hearing people, non-native speakers, and everyone in general. 

If the conference is in person, the venue must be accessible for people who use wheelchairs, have important signs in Braille, and a sound system compatible with hearing devices, just to name a few important features. Presenters should always speak into a microphone. Take care proactively of invisible disabilities, for example, by providing quiet spaces for privacy and noise-free conversations, menus that include sugar-free, gluten-free options. 
%ast: I removed the captioners part because it goes slightly off-topic and the sound system is apart for them when in person. 
Regardless of the conference format --online, in person, or hybrid-- all images used in the communication strategy of the conference should have alternative text, including the website. Likewise, any videos should have both captions and sound.

Platforms for conference registration and abstract submission, websites, and chat platforms --if used-- should be screen-reader friendly and keyboard accessible, with low technology requirements (hardware, software, and internet connection). 
Captioning or interpretation for presentations should be available in more than one language if possible. Sign languages are more complicated, because the vary from region to region, but if your conference has a clearly regional scope, a specific sign language (such as ASL, American Sign Language, or LIBRAS, the Brazilian Sign Language) can be adopted as official.

Do not forget about making your social events accessible and making sure that the materials presented are accessible for all. This includes physical activities that should not restrict participation based on body type or ability.

These aspects should be tested well in advance of going live.  

%teach about accessibility
Some practices, such as making accessible slides and presentations, are not yet common practice and will require great efforts from the presenters if they are not used to them. 
For that reason, the organizing team should provide accessibility guidelines for slides and presentations, encourage their use, and be available for any questions they may have. Among other things, the guidelines should ask for raw and accessible material to the talks before the conference, in screen-reader-frendly formats (e.g., R Markdown, HTML, or \TeX{} files). Accessibility in slides includes alternative text, and written code instead of print screen. 

 If presentations are pre-recorded, the speakers should include their video and ensure that their face is visible so that deaf and hard-of-hearing people can read their lips if needed (see \url{https://user2021.r-project.org/participation/accessibility/} for example). 
 
Most importantly, accessibility practices are not afterthoughts that can be dealt with at the last minute. 
They require time and early decision-making \cite{irishIncreasingParticipationUsing2020}, (see Rule \ref{rule_financial} about allocating resources). Conversely, inaccessible decisions are hard to course-correct, e.g., when finding out too late that a venue is inaccessible for people who use wheelchairs. 



\section{Don't let language restrict high-quality participation}
\label{rule_language}

In international conferences, English is often the official language. Submissions, presentations, tutorials, and workshops are in English. The platforms, the webpage, and official communications are also in English. While English is the primary language in scientific communication and one official language makes it conducive to communicate widely, opening up the conference to other languages could make it less intimidating to people who are not fluent in English \cite{ninerBetterWhomLeveling2021}. Excluding them may potentially lead to missing innovative contributions due to a language barrier. 


% Liz: People with disabilities don't want to be considered "needy". We reject the term "special needs" in part because it frames the person with the disability as being the burden causing the accommodation, where maybe it's the responsibility of society to be open to everyone. So I wonder if it's the same for people who speak other languages. Maybe they are not needy or inferior.  Or maybe it's wrong for me to apply disability rights philosophy to language.
The linguistic diversity of conference participants is often overlooked, resulting in missed opportunities for them, as well as for the conference to benefit from their contributions. Non-native English speakers may feel intimidated to ask questions or raise their views. Providing a welcoming and diverse environment by encouraging the full participation of those individuals is critical (Rule 2). If the panelist or speakers will not be presenting in their native language, it may helpful to rephrase audience questions in a less complex language. Consider letting speakers know in advance that they can ask for a question to be translated to them in their native language and that they can respond in their native language.

Advertising the conference in several languages and considering having non-English workshops and presentations (with or without captions in English) could help overcome this barrier. For instance, hosting one international day/session per conference might be a great place to start!

%ast: I think ome ideas of the first and second paragraphs can be joined because some parts are in both (intimidation, the importance of not missing opportunities) and maybe some practical tips can be made? Do we add an invitation to "bear" with subtitles and other languages to the English monolingual audience? an invitation to be respectful of "bad English", from submissions to presentations? 
%in person: live interpretation is doable. online: translated captions, pre-recorded talks, english trancripts

\section{Express the welcoming spirit in your communication strategy}
\label{rule_communication}
%ast: sara mentioned something that can be good spirit for the communication rule: don't do anything to the outside that you are not doing for your team. it's communication strategy but also there should be coherence between your actions and your words. %Rocío: I love that. Please add it.
An inclusive communication is to give each individual within the conference the opportunity to express themselves and a sense of belonging. Embedding inclusion in your communication strategy can only come by design, it should not be a thoughtless default. In the essence of your communication strategy, there should be coherence between actions and words. Always attend to the needs of your team, in other words, don't do anything to the outside that you are not doing for your team. 

As you design the promotion plan, actively reach out and promote the conference to people who have been systematically excluded. If part of the community has been historically discriminated against, emphasize that they are particularly welcome in this event and that the organization will make it a safe and inclusive environment. This could be done through publishing the diversity statement (e.g., https://user2021.r-project.org/about/diversitystatement/), the code of conduct, and accessibility guidelines [*here] and advertising support for attendees, such as fee waivers and financial support. People who fear getting rejected when applying for scholarships and waivers may find it relieving that the process will be supportive rather than discriminatory. 
Also, try to advertise the conference in multiple languages (\textbf{Rule \ref{rule_language}}) and express this welcoming spirit in your communication strategy (social networks, website, brand, and visual identity) to let people know that they are seen, respected, and welcome; that this is their space and community too.

Always include explicit descriptions of accessibility and other inclusion practices (\textbf{Rule \ref{rule_accessibility}}). Help people answer the question "Can I attend this conference?" without the emotional labor of contacting the organizers and be public about your commitment to equity and inclusion.%ast this explicit description seems to belong to the previous paragraph, maybe where I marked [*here]?
Be careful with the language you use in promoting the conference and how you talk about minoritized people, assertively set tone of inclusion. In addition to excluding derogatory or discriminatory language, make the effort to teach yourself the vocabulary and the best ways to communicate to account for every culture and situation. Do not expect people to teach you --it's not their role-- and accept feedback without being offended.

As we mentioned previously, conference attendees should have ample opportunities to network with each other. However, having a diverse offer of networking opportunities that can appeal to people with different backgrounds, accessibility needs, and preferences can be challenging, especially if you have the mindset of organizing each activity for the complete pool of attendants. If you think of smaller activities that can reach specific groups, preferably co-led by community leaders, these sessions can be more productive successful, and inclusive than trying to organize one single activity that can please the whole community (\textbf{Rule \ref{rule_unbias}}). Some examples are the newbies sessions for first-timers, mixers lead by specific subgroups or communities, or leisure activities that reunite subgroups with the same interests: arts, exercise, sports, movies, etc.
%Adithi had said: "Assist newbies of the conference navigate through the conference as they might feel overwhelmed attending a conference." but i see batool included newbies here :) 

% Yani: We need to work a little more here, perhaps with some examples.
% Rocío: could add some ideas from here 
% https://www.science.org.au/files/userfiles/support/emcr/documents/one-page-summary-emcr-improving-diversity-web.pdf
% Batool: I will work on this rule to include more details by 25th of Sept 
% ast: Thanks Batool! i added a couple of comments. 

\section{Allocate financial resources to support your conference goals}
\label{rule_financial}


Conference budgets are limited and rely mostly on sponsorship and attendance fees. Some expenses are more or less fixed but allocation of resources has to be intentional to support the goals of the conference. It is important to estimate the costs for these inclusive practices and define your priorities in advance. This includes paying at least part of your organizing team (see \textbf{Rule \ref{rule_organizing_team}}), training (e.g., code of conduct, active bystandership, \textbf{Rule \ref{rule_inclusion}}), and accessibility practices (\textbf{Rule \ref{rule_accessibility}}). It also includes additional support for attendees: child care support, transportation fees, visa-related support (if in-person), internet connection services (for the virtual component). While some of these items are being implemented recently, others can encounter a certain degree of resistance, such as paying the organizing team. Don't be afraid to innovate and resist the institutional inertia. "We have never payed for this, and this has always been like that" will not take the conference towards structural change.
%ast should we talk about the pandemic? something like: 
With the COVID-19 pandemic and the shift to online (or hybrid) events, some expenses disappeared, leaving space for rearrangements in this sense.
Depending on registration fees to support large budget items is not at all desirable, instead sponsors can be asked to support specific items in this list.


Registration costs are one of the largest barriers for conference attendance, and, if we are aiming for inclusiveness and representation, the socioeconomic context of participants, their country of origin, and their career status should be taken into account when determining the registration rates. Usually, there is a higher fee for people from the industry than for academia. A lower fee for non-profit organizations, government employees, or freelancers should also be considered. It is important to include discounts for students \cite{sarabipourChangingScientificMeetings2021, andalibPostdocQueueLabour2018, kaplanPostdocNot2012} and other early career stages, such as postdoc or trainee. The definition of such positions and the associated work conditions vary for each country. Consider that, while it is traditional for employers to provide--at least some--conference support in the Global North, this is not the case everywhere. In addition to this, low rates in strong currencies can represent significant amounts of money in other currencies. You can adjust the registration fee by the cost of living in each country using conversion factors (e.g., from the International Comparison Program report of the World Bank \cite{arendDisparityConferenceRegistration2019}, check https://spcanelon.github.io/useR2021-cost-conversion-tool/ for an example). In general, people should be allowed to locate themselves in a category they consider affordable, even with the possibility of a "pay what you can" approach. Resources permitting, you can aim to have a conference with no registration costs, but bear in mind that free events have a lower attendance rate than events with registration costs \cite{eventbrite_ultimate_2017}. Offering fee waivers to part of the participants is also a good option, but even then, other costs can be prohibitive. 

Offering scholarships or grants to attend the conference is a common way to boost participation from people from marginalized groups, and some conferences ask for cover letters or applications to them. These programs are important for in-person participation, which is more expensive than online participation, to support travel and lodging expenses. For online conferences, granting fee waivers is easier, and diversity scholarships can be replaced with broader participation support.
However, these applications may be counterproductive if they attract people that do not need the money but apply to enrich their CVs, and filter out people who do not feel entitled to earn them. Applying for loans, grants, and scholarships is an emotionally demanding task. Simplify applications as much as you can and be specific about the target audience of these programs. The broader the support, the simpler the application.



\section{Diversity and inclusion are processes}
\label{rule_process}

%Sara: I think the last rule should be about the process and reinforce to actually discuss and include diversity as a principle and not as a checklist (as it is on the text). Maybe also something about accountability and permanence. I think that this rule go be together with the rule_organizing_team given that it is relative to the team work and their dynamics.
%Sara: copying Yani comment here -  I like the "accountable and permanent" name. I think that an idea about progressive improvement can be mention here, like improve something every edition and no go back on the conquers goals. Now, to be accountable, you need to be transparent, and you need a clear structure; you get that with the structure of your team, with your declaration and guidelines, and with your action to reinforce those declarations and guidelines (if you have only declarations then you have only empty words, that sound good but you no need to comply. We discuss a lot about this).  For me, one more time, putting together a diverse team is what makes the other thing happened. We can cite The tyranny of structurelessness. The Second Wave, 2(1), 1972. on this rule. Perhaps we can talk about governance.
%ast: i'll put the last version of the tenth rule just to have an idea of how we left this
%past names: Have a bold vision but be ready to compromise when necessary/Aim for progress not perfection
%more people rules that have a continuity/accountability theme
%Aim for Progress, not Perfection
%Plan, balance and keep working to improve inclusion in the next edition of the conference!
%Assess whether equity and inclusion goals were met during the meeting
%Be accountable and permanent
%Use feedback from previous conferences or other conferences in your field to identify barriers to inclusion and plan around them

This rule is a complement to all the other rules. It is important to have a bold vision and clear targets. Some parts might be defined as baseline assumptions that are not up for discussion. However, there will always be compromises. Ideal solutions that meet all our expectations are often not possible. Don't let the perfect be the enemy of good. As a team, this is a challenging journey. Very early on, we tried to recognize that our resources were limited (both in time of the organizing team but also in money). We had to let go of certain aspects that were important to many of us. For an inclusive conference, we need to avoid giving in because considering all aspects seems impossible and burning ourselves and the team out because considering all aspects is actually impossible.


\section*{Concluding remarks}

The ten rules stated here can be adapted depending on the conference format and settings.
When organizing useR! 2021, we engaged in most of the practices mentioned here, and learned others along the way, so we share them here as part of our learning process. 
We organized useR! during a global pandemic, and as a team, this was a challenging journey. 
Very early on, we tried to recognize that our resources were limited (both in time of the organizing team and financially). 
We had to let go of certain aspects that were important to many of us. 
From that experience, this is our last message: your conference may not become perfectly inclusive and accessible, but the changes you make will make a difference.
Also, keep working to improve inclusion in the next edition of the conference! to document and balance your work, design surveys, ask for feedback from attendees and organizers. Transfer your constructed knowledge to the next organizing team, and adapt year-to-year to include more dimensions of diversity, to address technical and social changes, and to include insights from the new organizing team. Always remember inclusion is an ambition that organizers can and should continually pursue by allowing room for learning and improvement.
If more conferences and domains apply these rules, the process will get more streamlined, straightforward, and mainstream to adapt with minimal overhead.
And you will make a change towards healthier, stronger, and more inclusive communities.


% Try not to burn you and the team out trying to make it perfect.  


\section*{Acknowledgments}
The authors of this piece would like to thank every single member of the organizing team of useR! 2021 [ \url{https://user2021.r-project.org/about/global-team/}] for their valuable contribution to an inclusive conference experience, and the R Foundation for charging us with the organization of useR! 2021 and supporting us through the process. 


% \nolinenumbers

% % Either type in your references using
% % \begin{thebibliography}{}
% % \bibitem{}
% % Text
% % \end{thebibliography}
% %
% % or
% %
% % Compile your BiBTeX database using our plos2015.bst
\bibliography{community-science}
% % style file and paste the contents of your .bbl file
% % here. See http://journals.plos.org/plosone/s/latex for 
% % step-by-step instructions.
% % % 
% \begin{thebibliography}{10}

% \bibitem{bib1}
% Conant GC, Wolfe KH.
% \newblock {{T}urning a hobby into a job: how duplicated genes find new
%   functions}.
% \newblock Nat Rev Genet. 2008 Dec;9(12):938--950.

% \bibitem{bib2}
% Ohno S.
% \newblock Evolution by gene duplication.
% \newblock London: George Alien \& Unwin Ltd. Berlin, Heidelberg and New York:
%   Springer-Verlag.; 1970.

% \bibitem{bib3}
% Magwire MM, Bayer F, Webster CL, Cao C, Jiggins FM.
% \newblock {{S}uccessive increases in the resistance of {D}rosophila to viral
%   infection through a transposon insertion followed by a {D}uplication}.
% \newblock PLoS Genet. 2011 Oct;7(10):e1002337.

% \end{thebibliography}



\end{document}


% Template for PLoS
% Version 3.5 March 2018
%
% % % % % % % % % % % % % % % % % % % % % %
%
% -- IMPORTANT NOTE
%
% This template contains comments intended 
% to minimize problems and delays during our production 
% process. Please follow the template instructions
% whenever possible.
%
% % % % % % % % % % % % % % % % % % % % % % % 
%
% Once your paper is accepted for publication, 
% PLEASE REMOVE ALL TRACKED CHANGES in this file 
% and leave only the final text of your manuscript. 
% PLOS recommends the use of latexdiff to track changes during review, as this will help to maintain a clean tex file.
% Visit https://www.ctan.org/pkg/latexdiff?lang=en for info or contact us at latex@plos.org.
%
%
% There are no restrictions on package use within the LaTeX files except that 
% no packages listed in the template may be deleted.
%
% Please do not include colors or graphics in the text.
%
% The manuscript LaTeX source should be contained within a single file (do not use \input, \externaldocument, or similar commands).
%
% % % % % % % % % % % % % % % % % % % % % % %
%
% -- FIGURES AND TABLES
%
% Please include tables/figure captions directly after the paragraph where they are first cited in the text.
%
% DO NOT INCLUDE GRAPHICS IN YOUR MANUSCRIPT
% - Figures should be uploaded separately from your manuscript file. 
% - Figures generated using LaTeX should be extracted and removed from the PDF before submission. 
% - Figures containing multiple panels/subfigures must be combined into one image file before submission.
% For figure citations, please use "Fig" instead of "Figure".
% See http://journals.plos.org/plosone/s/figures for PLOS figure guidelines.
%
% Tables should be cell-based and may not contain:
% - spacing/line breaks within cells to alter layout or alignment
% - do not nest tabular environments (no tabular environments within tabular environments)
% - no graphics or colored text (cell background color/shading OK)
% See http://journals.plos.org/plosone/s/tables for table guidelines.
%
% For tables that exceed the width of the text column, use the adjustwidth environment as illustrated in the example table in text below.
%
% % % % % % % % % % % % % % % % % % % % % % % %
%
% -- EQUATIONS, MATH SYMBOLS, SUBSCRIPTS, AND SUPERSCRIPTS
%
% IMPORTANT
% Below are a few tips to help format your equations and other special characters according to our specifications. For more tips to help reduce the possibility of formatting errors during conversion, please see our LaTeX guidelines at http://journals.plos.org/plosone/s/latex
%
% For inline equations, please be sure to include all portions of an equation in the math environment.  For example, x$^2$ is incorrect; this should be formatted as $x^2$ (or $\mathrm{x}^2$ if the romanized font is desired).
%
% Do not include text that is not math in the math environment. For example, CO2 should be written as CO\textsubscript{2} instead of CO$_2$.
%
% Please add line breaks to long display equations when possible in order to fit size of the column. 
%
% For inline equations, please do not include punctuation (commas, etc) within the math environment unless this is part of the equation.
%
% When adding superscript or subscripts outside of brackets/braces, please group using {}.  For example, change "[U(D,E,\gamma)]^2" to "{[U(D,E,\gamma)]}^2". 
%
% Do not use \cal for caligraphic font.  Instead, use \mathcal{}
%
% % % % % % % % % % % % % % % % % % % % % % % % 
%
% Please contact latex@plos.org with any questions.
%
% % % % % % % % % % % % % % % % % % % % % % % %

\documentclass[10pt,letterpaper]{article}
\usepackage[top=0.85in,left=2.75in,footskip=0.75in]{geometry}

% amsmath and amssymb packages, useful for mathematical formulas and symbols
\usepackage{amsmath,amssymb}

% Use adjustwidth environment to exceed column width (see example table in text)
\usepackage{changepage}

% Use Unicode characters when possible
\usepackage[utf8x]{inputenc}

% textcomp package and marvosym package for additional characters
\usepackage{textcomp,marvosym}

% cite package, to clean up citations in the main text. Do not remove.
\usepackage{cite}

% Use nameref to cite supporting information files (see Supporting Information section for more info)
\usepackage{nameref,hyperref}

% line numbers
\usepackage[right]{lineno}

% ligatures disabled
\usepackage{microtype}
\DisableLigatures[f]{encoding = *, family = * }

% color can be used to apply background shading to table cells only
\usepackage[table,dvipsnames]{xcolor}

% array package and thick rules for tables
\usepackage{array}

%strikethrough
\usepackage{soul}

%sara: for commenting the pdf to add alt text (if anyone knows a better solution for atl text please help)
\usepackage{pdfcomment}

% create "+" rule type for thick vertical lines
\newcolumntype{+}{!{\vrule width 2pt}}

% create \thickcline for thick horizontal lines of variable length
\newlength\savedwidth
\newcommand\thickcline[1]{%
  \noalign{\global\savedwidth\arrayrulewidth\global\arrayrulewidth 2pt}%
  \cline{#1}%
  \noalign{\vskip\arrayrulewidth}%
  \noalign{\global\arrayrulewidth\savedwidth}%
}

% \thickhline command for thick horizontal lines that span the table
\newcommand\thickhline{\noalign{\global\savedwidth\arrayrulewidth\global\arrayrulewidth 2pt}%
\hline
\noalign{\global\arrayrulewidth\savedwidth}}


% Remove comment for double spacing
%\usepackage{setspace} 
%\doublespacing

% Text layout
\raggedright
\setlength{\parindent}{0.5cm}
\textwidth 5.25in 
\textheight 8.75in

% Bold the 'Figure #' in the caption and separate it from the title/caption with a period
% Captions will be left justified
\usepackage[aboveskip=1pt,labelfont=bf,labelsep=period,justification=raggedright,singlelinecheck=off]{caption}
\renewcommand{\figurename}{Fig}

% Use the PLoS provided BiBTeX style
\bibliographystyle{plos2015}

% Remove brackets from numbering in List of References
\makeatletter
\renewcommand{\@biblabel}[1]{\quad#1.}
\makeatother



% Header and Footer with logo
\usepackage{lastpage,fancyhdr,graphicx}
\usepackage{epstopdf}
%\pagestyle{myheadings}
\pagestyle{fancy}
\fancyhf{}
%\setlength{\headheight}{27.023pt}
%\lhead{\includegraphics[width=2.0in]{PLOS-submission.eps}}
\rfoot{\thepage/\pageref{LastPage}}
\renewcommand{\headrulewidth}{0pt}
\renewcommand{\footrule}{\hrule height 2pt \vspace{2mm}}
\fancyheadoffset[L]{2.25in}
\fancyfootoffset[L]{2.25in}
\lfoot{\today}

%% Include all macros below

\newcommand{\lorem}{{\bf LOREM}}
\newcommand{\ipsum}{{\bf IPSUM}}

%% END MACROS SECTION


\begin{document}

\newcommand{\fede}[1]{\textcolor{ForestGreen}{Fede: #1}}
\newcommand{\rocio}[1]{\textcolor{MidnightBlue}{Rocio: #1}}
\newcommand{\doro}[1]{\textcolor{RedViolet}{Dorothea: #1}}
\newcommand{\jani}[1]{\textcolor{YellowOrange}{Janina: #1}}
\newcommand{\la}[1]{\textcolor{RawSienna}{Laura: #1}}
\newcommand{\as}[1]{\textcolor{Violet}{Andrea: #1}}

\vspace*{0.2in}

% Title must be 250 characters or less.
\begin{flushleft}
{\Large
\textbf\newline{Ten simple rules towards hosting an inclusive conference} % Please use "sentence case" for title and headings (capitalize only the first word in a title (or heading), the first word in a subtitle (or subheading), and any proper nouns).
}
\newline
% Insert author names, affiliations and corresponding author email (do not include titles, positions, or degrees).
\\
Name1 Surname\textsuperscript{1,2\Yinyang},
Name2 Surname\textsuperscript{2\Yinyang},
Name3 Surname\textsuperscript{2,3\textcurrency},
Name4 Surname\textsuperscript{2},
Name5 Surname\textsuperscript{2\ddag},
Name6 Surname\textsuperscript{2\ddag},
Name7 Surname\textsuperscript{1,2,3*},
with the Lorem Ipsum Consortium\textsuperscript{\textpilcrow}
\\
\bigskip
\textbf{1} Affiliation Dept/Program/Center, Institution Name, City, State, Country
\\
\textbf{2} Affiliation Dept/Program/Center, Institution Name, City, State, Country
\\
\textbf{3} Affiliation Dept/Program/Center, Institution Name, City, State, Country
\\
\bigskip

% Insert additional author notes using the symbols described below. Insert symbol callouts after author names as necessary.
% 
% Remove or comment out the author notes below if they aren't used.
%
% Primary Equal Contribution Note
\Yinyang These authors contributed equally to this work.

% Additional Equal Contribution Note
% Also use this double-dagger symbol for special authorship notes, such as senior authorship.
\ddag These authors also contributed equally to this work.

% Current address notes
\textcurrency Current Address: Dept/Program/Center, Institution Name, City, State, Country % change symbol to "\textcurrency a" if more than one current address note
% \textcurrency b Insert second current address 
% \textcurrency c Insert third current address

% Deceased author note
\dag Deceased

% Group/Consortium Author Note
\textpilcrow Membership list can be found in the Acknowledgments section.

% Use the asterisk to denote corresponding authorship and provide email address in note below.
* correspondingauthor@institute.edu

\end{flushleft}
% Please keep the abstract below 300 words
\section*{Abstract (optional from what I've seen)}

The authors of this article participated in the organization of the annual conference of the R Project for Statistical Computing, held in July 2021. useR! conferences are non-profit events organized by volunteers from the R community and managed by the R Foundation. The conference attracts a broad range of participants from academia, industry, government, and the non-profit sector. For 2021, we aimed to build a high-quality virtual and explicitly global conference in a kind, inclusive, accessible, and welcoming environment for everyone. 
In this article, we outline our most important lessons learned in 10 simple rules to host an inclusive conference.\fede{general comment on this: we should clearly state that this whole-virtual setting was a clear consequence of the pandemics, AND it was planned like this from the beginning, and not changing in-between to this format} These rules apply equally to academic, industry, or mixed conferences; the rules are inspired by a global experience but also apply at the regional or local level. The rules were learned during a virtual conference, but serve also as guidelines to in-person and hybrid events. 

%andrea: matt's original sentence: reaching users and developers of the R language from more than 120 countries. 



% % Please keep the Author Summary between 150 and 200 words

\linenumbers

\section*{Introduction}

% Rocío: Main message of the paragraph: Why we are writing this piece
Conferences, from the Latin \textit{conferre}--`to bring together', are spaces to meet and reconnect with members from a specific community, learn about advances in the field, and share our recent contributions.\fede{between this and the next sentence, there is a strong break. need some ligation, something like "Despite the many efforts in this direction, ..."}
However, conferences are likely to reproduce the systematic discrimination occurring in other spaces in our fields. 
Lack of representation and unwelcoming--or overtly aggressive--environments exclude people from opportunities for learning, sharing, and producing knowledge, and becoming active community members, all of which are readily provided to non-marginalized people \cite{hendersonThoughtfulGatheringsGendering2020}.
Exclusionary conference experiences can divert career paths, affect lives, and drive people out of academia \cite{biggsAcademicConferenceChilly2018, hendersonThoughtfulGatheringsGendering2020}. 
This message is not new. 
Barriers such as unattainable registration costs, sexism, \as{prejudice against disable people ( ableism)}, among others, have already been exposed and discussed in the literature \cite{arendDisparityConferenceRegistration2019, biggsAcademicConferenceChilly2018, depickerRethinkingInclusionDisability2020a, irishIncreasingParticipationUsing2020}, 
and some proposals for more inclusive conferences have been put into practice \cite{gichoraTenSimpleRules2010a, levitisCenteringInclusivityDesign2021, atkinsonJournalMedicine20202021, foramittiVirtuesVirtualConferences2021, ninerBetterWhomLeveling2021, rabyMovingAcademicConferences2021, noauthor_discover2021}, with a primary focus on the online format to open doors for inclusion.\fede{do we need maybe a short definition of the concepts of diversity-equity-inclusion for us/refer to something?} %LA: https://www.cscce.org/resources/organizing-community-events/


% Rocío: Main message of the paragraph: Roadmap of the paper
This article focuses on practices for hosting inclusive conferences.
The rules written here are directed to people who are part of a stable meetings committee that oversees the site/location selection process, or that coordinates with the local organizers of conferences.
The rules can also be helpful to local/virtual organizers who intend to put inclusion at the core of the planning phase.
These tips stem from the authors' experience of organizing useR! 2021, a virtual and global statistical computing conference for users and developers of the R programming language \cite{r_core_team_2021}. 
We embraced the challenge of organizing a high-quality virtual conference in the context of the COVID-19 pandemic and making it a kind, inclusive, and accessible experience for as many persons as possible. 
Here, we share the lessons learned within the past year of organizing useR! 2021, summarized as ten simple rules towards hosting an inclusive conference.
The rules are organized in three groups (Figure \ref{fig:diagram}). %original source: https://docs.google.com/drawings/d/1iS1pLc9OldLMe_jhNT6OGIbT9s5KvlHpUoDd19vHkCY/edit?usp=sharing
%sara: still need to better format figure size and get the alt text right :P
Group 1 includes rules 1, 2, and 10, which refer to pillars of an inclusive conference: embracing diversity in all its dimensions, creating a safe and welcoming environment for everyone, and making the conference part of a long-term process for inclusion.
The second group includes rules 3 and 4 and focuses on the people who participate in the conference. 
Rule 3 refers to the importance of working with an inclusive and diverse organizing team, and Rule 4 concerns the necessity of counteracting implicit and systemic bias from spotlight roles like keynote speakers, other presenters, program committee members, or other session chairs. 
The third group includes rules 5 to 9. These rules are about components of the conference that should be carefully planned for: an online component, accessibility for people with disabilities, language inclusiveness, a welcoming communication strategy, and financial resources to support inclusion. 
These rules apply equally to conferences where the expected audience is academic, from the industry, or a mixed group. While these rules are inspired in a global virtual experience they also apply at the regional or local level and can inform in-person and hybrid events.
\fede{maybe something like: "Even if this set of rules is derived from our experience with a global scale event, we believe these can be easily transferred to events with regional or local scopes"}

\begin{figure}[!h]
\centering
%sara: without success i'm trying to use pdfcomment package and pdftooltip to make the alt-text. any help?
\pdftooltip{
\includegraphics[width=\textwidth]{figs/10_rules_diversity.png}}{
%sara: suggestion for alt text:
Diagram of how the 10 rules are organized into three groups: grounding rules (rules 1, 2, and 10), people-related rules (rules 3 and 4), and design rules (rules 5 to 9). The diagram has five rows. Each line contains rectangular boxes with the rule number and a short title for each rule. Each rule box is colored based on their group (grounding rules are grey, people-related rules are green and design rules are pink). First row is for the grounding rules 1. Multidimensional diversity and 2. Safe and inclusive environment in grey boxes. Second row is for people-related rules 3. Organizing team and 4. Unbias in green boxes. Third row is for the design rules 5. Online component, 6. Accessibility for disabilities, 7. Language inclusivity in pink boxes. The fourth row is for the design rules 8. Communication and 9. Financial support and budget in pink boxes. The last row is for the grounding rule 10. Diversity and inclusion as a process in grey boxes.
}
\caption{Diagram of the rules organized in three groups: grounding rules (rules 1, 2, and 10), people-related rules (rules 3 and 4), and design rules (rules 5 to 9).}
\label{fig:diagram}
\end{figure}

% Dorothea: I like the intro as it is now, and I don't think the following paragraph is necessary.
% Rocío: I'm leaving this out now and seeing if we don't need this description of communities
%% Rocío: Main message of the paragraph: We do this for communities and people. What do we mean by communities and why are they important? (Why are they important may need some text)
%This article suggests rules to pivot traditional conferences towards diversity and inclusion, and strive to build more inclusive and welcoming communities. 
%Some scholarly or technological communities are formally structured as learned societies (e.g. the International Geographical Union or the Royal Statistical Society), others are composed by networks of less formal local groups (e.g. R User Groups or Python User Groups), and others may not have any established structure. 
%For the latter, conferences may play even a bigger role in shaping the community, since there is no other organization or institution to gather the community members. 

% keeping references:
%   However, conferences can become discriminating spaces, in which members of some specific privileged groups reproduce the systematic inequalities that occur in academia and society \cite{arendDisparityConferenceRegistration2019, timperleyHeMoanaPukepuke2020, gewinWhatScientistsShould2019, brownAbleismAcademiaWhere2018, marks2021meeting}.


\section{Embrace all dimensions of diversity}
\label{rule_diversity}

% Rocío: Main idea: what is diversity and why does it matter? (inequalities)
Diversity encompasses multiple dimensions: age, physical ability, career stage, gender, gender identity, geographic origin, language, neurodiversity, race, religion, sexual orientation, and socioeconomic background, to name a few.
Human diversity should be celebrated and respected in every way. Nonetheless, we live in a world with implicit hierarchies along these axes. Some statuses (e.g., cisgender, white, male, from the United States or Western Europe) hold the privilege of being the defaults around which all systems--including conferences--are consciously and unconsciously built. \as{People outside these groups suffer different forms of oppression, like sexism, racism, ableism, homophobia, and transphobia. Furthermore, no diversity dimension operates in isolation from the others, they intersect in complex ways to determine people's experience in the world--a concept know as \textit{intersectionality}--so that anyone can suffer from some forms of oppression and be privileged along some other dimension, and some groups are particularly vulnerable and systematically left behind.}
While no isolated initiative can change reality by itself, building a more diverse and inclusive conference starts by recognizing that these inequalities have systematically excluded whole groups of people from academia and scientific and professional circles \cite{timperleyHeMoanaPukepuke2020}. 

% Rocío: more effort towards including the more excluded
Recognizing our privileges--unearned advantages given by society to some people but not all--particularly in our field and in our scientific or professional community, will help identify which subgroups have been the most excluded or discriminated against. 
These are the groups we need to make more effort to include. \fede{General thought: I like the way we say WE and YOU. I still think I saw most such 10SR papers speak to the readers addressing them with the roles. Which would be "Organizers", "Attendees". Just my 2 cents as avid reader of 10SR papers ;)}
Investing more effort towards the most excluded groups does not mean neglecting the others, but it will guide the vision of diversity for your conference--and your strategies to achieve it.
Will a more diverse conference translate into a gender distribution of your speakers that is representative of the general population? 
Would it be the presence of racialized people --especially Black people-- among the head organizers, speakers, and attendees? 
Would it be having LGBTQIA+ friendly-spaces or community participation from key geographic regions?
The answer to these questions depend on your field, region, or community.


%batool: There are a few references which define privilege and might be good to add here including: Friedman, S., O’Brien, D., & McDonald, I. (2021). Deflecting Privilege: Class Identity and the Intergenerational Self. Sociology, 55(4), 716–733. https://doi.org/10.1177/0038038520982225
%batool: Another one is: Crevani, L. (2019). Privilege in place: How organisational practices contribute to meshing privilege in place. Scandinavian Journal of Management, 35(2). https://doi.org/10.1016/j.scaman.2018.09.002
%batool: please note that I didn't read both entire papers I mentioned above.
%sara: reference for privilege: McIntosh, Peggy. "White privilege: Unpacking the invisible knapsack." (1988).
% Rocío: I feel like we don't need to cite any reference about privilege the way it is now. But feel free to disagree.
%Dorothea: I agree with Rocío. citations for priviledge feel like citation for gravity...the current wording is not saying "there is priviledge, as shown by xyz. it says there is priviledge and we need to see where it is particularly strong in our field.


\section{Create a safe and welcoming environment}
\label{rule_inclusion}

% Rocío: main idea: Inclusion for everyone
While it is essential to improve representation towards some of the most visible dimensions of human diversity, such as race, gender, and country of origin, building a truly inclusive environment means taking care of all the other aspects of diversity as well. 
Accommodating for religious practices, setting specific accommodations for breastfeeding and child care, scheduling events exclusive for diverse groups (e.g., LGBTQIA+-friendly spaces), promoting and respecting pronouns for all genders and having gender-neutral bathrooms, are just some examples of decisions that are relatively easy to implement and can have a large impact in making inclusion real (see \cite{noauthor_discover2021} for some great advice on `low-hanging fruits').
Importantly, by being proactive in creating such a welcoming space, you can be respectful of everybody's privacy and avoid requiring anyone to disclose personal information–such as revealing a disability, the sexual orientation, gender identity, or mental health issues, just to a name a few.

% Rocío: main idea: Code of conduct for a safe place
Adopting a Code of Conduct and gathering a team to enforce it are key aspects in creating a safe environment during a conference \cite{favaroYourScienceConference2016}.
The Code of Conduct is a document meant to keep the community safe and should state clearly the unacceptable behaviors, the consequences for engaging in such behavior, and the way to report violations \cite{auroraHowRespondCode2019}. 
The Code of Conduct team must receive training on how to receive reports, respond to incidents, communicate their responses, and organize accordingly. 
Since people from marginalized groups are more likely to be victims of Code of Conduct violations and to see their claims dismissed due to power dynamics, assembling a diverse Code of Conduct team will be essential to deal with discrimination and harassment issues and the power dynamics underneath them and deal with them in a sensitive, victim-centered manner. It may also be more understanding of different disciplinary cultures and geographical and cultural considerations. 
We strongly recommend reading `How to Respond to Code of Conduct Reports' \cite{auroraHowRespondCode2019} as an excellent starting point for the Code of Conduct teamwork.
% Rocío: two of these sentences, diverse CoC team, and communication of CoC, may belong to rules organizing team and communication, respectively. 

% Rocío: Could have a paragraph about active bystandership—Andrea%ö

\section{Have an inclusive and diverse organizing team}
\label{rule_organizing_team}

% Rocío: Idea of this paragraph: build a diverse team, and representation

A genuinely inclusive conference can only be organized by an inclusive and diverse organizing team, in charge of the conference decision-making.
Organizers should build a team with people from different regions, genders, races and ethnicities, socioeconomic statuses, and other aspects of diversity.
Having diverse people in decision-making positions will affect positively all the other aspects of your conference because all the processes will benefit from their input, expertise, and distinct perspectives \cite{hongGroupsDiverseProblem2004}. 
In addition, a diverse team plays an important role at creating a welcoming space because representation--seeing people with similar life experiences occupy public spaces, positions of power, and breaking negative stereotypes--is one of the best ways to create a sense of belonging for everyone participating in the conference (see \textbf{Rule \ref{rule_inclusion}}). If you already started assembling an organizing team, check for gaps in its composition. 
The regional and local communities, groups, or associations in your field are good sources to tap into. 

% Rocío: Idea of this paragraph: non-transferability and do not restrict people
People with disabilities often say: `Nothing about us without us'; the same holds for other dimensions of diversity. This means that the actual life experiences, expertise, and insights from people in marginalized groups cannot be replaced by good intentions from people outside these groups \cite{costanzachockDesign2020}.
People who have experienced exclusion have the social and technical expertise to fight against it and they will be essential in any team.
However, they must have the freedom to choose in which areas of the conference they want to work and not be restricted to diversity and inclusion aspects. 

% Rocío: Idea of this paragraph: tips to make a diverse and inclusive team work + paying them
When working with people from marginalized groups do not assume that self-nomination and voting will work as mechanisms to choose leadership positions. Instead, nominate directly and offer leading positions that would normally be occupied by people from privileged groups.
Build an environment in which every person can express their position and give priority to people from systematically excluded groups.
Many people lack the institutional support to put time and effort into the organization tasks and do not have the luxury to commit to the organization for free; consider paying them as an item in your budget.  

% Saranjeet: At times regional institutes and universities do not support nor recognize the time put in organization tasks. As such one needs to carry out their day work for the institute while working extra hours for organizing. It might be challenging to find a way to recognize organizers that works internationally. However, there should certainly be some mechanism put into place which recognizes the time of the organizers.
In addition, tasks such as receiving and responding to Code of Conduct reports can be emotionally intense work and should be additionally rewarded.

% % Rocío: might go to the next rule
% Splitting the workload and responsibilities should not be done by putting care-taking labors--community building, meeting organization and note-taking, conversations with potential partners--on the hands of women and other minoritized groups, while people from privileged groups take the lead in stereotypical highly-valued tasks (see \textbf{Rule \ref{rule_unbias}}). 


\section{Consciously counteract bias in your spotlight roles}
% in the conference program
\label{rule_unbias}

%idea of paragraph 1: our lists are biased

When choosing or inviting people for visible and valued roles in the conference––such as keynote speakers, program committee members, session chairs––, it is likely that there will not be much diversity in the first set of names.
Our biased lists are products of the existing systems that have always privileged some groups of people \cite{dwyerNoticeWhoScience2021,swartzScienceValueDiversity2019,wongBuildDiversityScience2020,dignazioUnicornsJanitorsNinjas2020}. 
Rather than deter us, this should encourage us to go beyond our narrow and often limited networks to look for, reach out to, invite, encourage, and onboard great people that are not routinely in the spotlight, making sure that these roles do not stay with people in privileged groups (e.g. avoid `manels' and all-white panels, for instance).

Having a chance to present at a conference is an opportunity to gain--or keep--visibility in the community or field. 
Even diverse selection committees can be subjected to unconscious bias, by interpreting information such as the location of the institution or the names of the authors and attaching value to their work according to negative or positive stereotypes. 
Selection committees need to be reminded of such biases, and try to counteract them when evaluating submissions and presentations, which should have clear evaluation criteria.
\cite{swartzScienceValueDiversity2019, wongBuildDiversityScience2020}
 
% Rocío: main idea: unbias roles and give spotlight to the roles--and people who have contributed in roles--that are not stereotypically categorized as success


Activities such as scientific publication and development of software are often regarded as higher value than care-taking roles like community building or teaching. 
These roles may be equally if not more challenging and are usually assumed by women, people of color, people with disabilities, and other minoritized groups \cite{cheng2020x+, burfordHomelinessMeantHaving2020}. Research about these topics is equally undervalued compared to other areas of knowledge production. 
Counteracting bias also means broadening the range of topics of the conference and giving spotlight/visibility to usually undervalued research and the people behind it.
%Defy the stereotypical criteria for merit by acknowledging these community practices and the people behind them.
You can also reframe the awards ceremony to acknowledge those who contributed to community building, presented teaching materials, and assumed other care-taking roles.


\section{Have a strong online component} 
\label{rule_online}

% Rocío: main idea of the paragraph: online conferences can open opportunities for inclusion
In-person interaction at conferences is priceless but often this is true and possible only for the ones who can afford to attend. 
Barriers such as cost of registration, transport and accommodation, the logistics of long-distance travel, and discriminatory visa applications, are particularly true for conferences that usually take place in high-income countries \cite{arendDisparityConferenceRegistration2019,gewinWhatScientistsShould2019,jooKeepOnlineOption2021}. 
Online conferences are more inclusive because they do not require a visa or a big budget, and are more accessible to people who may be unable to travel because of health issues or family responsibilities \cite{salibaGettingGripsOnline2020}.
This means that online conferences have a greater reach, not only in terms of participants but in terms of the tutors and presenters that can participate \cite{atkinsonJournalMedicine20202021, roosOnlineConferencesNew2020, jooKeepOnlineOption2021}.
The online format may also make it easier to be inclusive of geographic regions by encompassing several timezones or deciding to rotate the favored timezone every year, without depending on a conference central location. 

%Saranjeet: About time zones, although we can rotate and have different days dedicated to different time zones, we could also try spreading the conference time to be longer. So that each day the conference runs live for a couple of hours or so and the recorded material is made available/uploaded on YouTube (say) soon (every day). This might help because, not everyone might have the luxury of attending 4-6 hours of the conference on a weekday, even though it is running in their local timezone. However, people might be able to attend a couple of hours even on weekdays and will not get the feeling that they have missed a huge chunk of the conference. 

% Saranjeet: Not to include in this paper, but a general suggestion. Why not select a platform which helps in live broadcasting of the conference on LinkedIn as well? For example, Streamyard (https://streamyard.com/) is one such multistreaming platform, that can simultaneously stream to YouTube, LinkedIn, Twitter, and other platforms

% Rocío: main idea of the paragraph: In the context of reopening, hybrid conferences may be a thing. They also need to be inclusive.
Within the context of the COVID-19 pandemic, many conferences embraced the online format; but at the time of writing, some are reverting to in-person, which risks going back to the barriers mentioned above \cite{jooKeepOnlineOption2021}.
Alternatively, a conference could have a hybrid format with an in-person and an online components \la{that articulate well into a hybrid format that maximizes the experience of everyone attending your conference}. This dual format could allow a group of people to interact face-to-face while providing many others the opportunity to participate remotely. 
The challenge and requirement for this kind of setting would be to make the online component as relevant as the in-person component and not just a consolation prize to the less privileged in the community \cite{ninerBetterWhomLeveling2021}.

In fact, experts in theorizing about and building communities of practice [Wenger, 1999] who have been holding hybrid events since 2011 [Wenger-Tayner, 2021] consider that inclusive hybrid events are the future. These authors emphasize the need that all participants of hybrid events must go through the same event quality to the point that in-person and online participants pay the same registration amount to participate in high-quality hybrid conferences. These authors recommend having a buddy system, where every person attending online has an in-person attendant that makes sure the online person can follow the event and participate in the event as if they were in the room [Wenger-Trayner, 2020]. Wenger-Trayner (2020, 2021) also recommends having a chat function for back-channel conversations between all attendants, a virtual space for all online attendants to talk to each other, and allocating time to small group conversations, often mixing in-person and online groups. The physical venue will have to include multiple ways to connect with online folks such as tablets on the meeting venue tables and cameras that allow online attendants to follow who is speaking in the in-person space. 
%an online component is not inclusive \textit{per se} without taking many other precautions

%For instance, to avoid favoring the same regions in every edition, we suggest rotating the location of the conference—if it is global, moving from one continent to another. 
% Rocío: Now I'm not sure if we need the last two sentences that I commented. 

% Rocío: main idea: Networking may seem the weakest part of online conference, but it doesn't have to be. 
% Rocío: I'm still doubting if this should be the 2nd paragraph, and I think that it needs some editing.
Networking and socializing have been mentioned as challenging aspects of online conferences, mostly because we are used to interactions at in-person social settings such as coffee or lunch breaks \cite{salibaGettingGripsOnline2020, roosOnlineConferencesNew2020}. 
However, in-person settings might not be comfortable or appealing to everyone for socializing and 
some of these spaces may be exclusionary (e.g. galas or dinner nights at expensive venues). 
On the other hand, there is evidence that virtual communication before face-to-face interaction can make people from minoritized groups feel more included, thus participate more (e.g. \cite{trianaDoesOrderFacetoFace2012, blackEngenderingBelongingThoughtful2020}). 
Organizers of online conferences should invest time in creating opportunities to meet and bond virtually, respecting people's limits, preferences, and remembering that `the usual' does not necessarily work for everyone, and that no single networking activity will ever serve the whole community. 
It is worth trying new and varied activities centered around different groups of people outside the mainstream of your conference %LA cita 
.
Some ideas in this line are: offering the option of written chat instead of voice or video conversations, opening events with teamwork like trivia, offer some events that can be enjoyed passively like movies, yoga sessions, or art displays, where attendants can choose to just sit and enjoy without talking, or have a chat channel to comment on their experiences during the session.\fede{another activity: virtual city tours, I had it as a corollary activity for a summer school} \la{Some more specifics on how to achieve these suggestions in this paper: https://osf.io/k3bfn/}


% we need to check if something here (newbies!) is not missing in the above paragraphs
% As we mentioned previously, conference attendees should have ample opportunities to network with each other. However, having a diverse offer of networking opportunities that can appeal to people with different backgrounds, accessibility needs, and preferences can be challenging, especially if you have the mindset of organizing each activity for the complete pool of attendants. If you think of smaller activities that can reach specific groups, preferably co-led by community leaders, these sessions can be more productive successful, and inclusive than trying to organize one single activity that can please the whole community (\textbf{Rule \ref{rule_unbias}}). Some examples are the newbies sessions for first-timers, mixers lead by specific subgroups or communities, or leisure activities that reunite subgroups with the same interests: arts, exercise, sports, movies, etc.
%Adithi had said: "Assist newbies of the conference navigate through the conference as they might feel overwhelmed attending a conference." but i see batool included newbies here :) 

\section{Make the conference accessible to people with disabilities}
\label{rule_accessibility}


Conferences are among the least accessible spaces that people with disabilities may encounter in professional contexts \cite{priceAccessImaginedConstruction2009}. Even when conferences implement other inclusive practices, the participation of people with disabilities is often overlooked \cite{marks2021meeting}. This is one key aspect where having someone in the organizing team allows them to take part in the decisions from the beginning (see \textbf{Rule \ref{rule_organizing_team}}), as thinking about disability or simulating it are not substitutes for real-life experience \cite{costanzachockDesign2020}. Planning for accessibility requires time and early decision-making \cite{irishIncreasingParticipationUsing2020}. As other aspects of inclusion, dealing with accessibility at the last minute is the recipe for a disastrous conference. If you do not consider accessibility from the conference inception, it is highly likely you will be better off without trying to patch accessibility at the last minute. Key decisions in this respect are hard to correct. 
For in-person conferences, the venue should comply with common accessibility standards, such as being adequate for people who use wheelchairs, have signs in Braille, and a sound system compatible with hearing devices and live interpretation, just to name a few important features. In addition to this, the organizers should take care proactively of invisible disabilities, for example, by providing quiet spaces for privacy and noise-free conversations, provide chairs in open spaces, and menus that menus that account for the dietary requirements participants need to respect.

Regardless of the conference format, all platforms (website, chat, conference administration tools) and images used for the communication strategy of the conference should be screen-reader friendly and keyboard accessible, and have alternative text. Any videos should have good-quality captions and a transcript.

The organizing team should provide accessibility guidelines for slides and presentations, encourage their use, and be available for any questions presenters and attendees may have. Slide decks should be made available beforehand, either in webpages or available for download to ensure that everybody can follow their content during the presentations. 

Accessibility practices should also include social events and networking, and include activities that do not restrict participation based on body type or ability. Importantly, all these accessibility practices are inclusive not only for people with disabilities but to everyone %ast: refs here.
For instance, captions are helpful for non-native speakers, having the material available for download helps attendees with low bandwidth connection, etc. Accessibility efforts should be explicitly displayed in the official communication channels managed by the conference organizers (see \textbf{Rule \ref{rule_communication}}). 




\section{Don't let language restrict high-quality participation}
\label{rule_language}

% Rocío: main idea: English as the only language makes some people privileged and is a barrier for others
In international meetings, the linguistic diversity of the participants is often overlooked. 
English is usually the official and sole language for submissions, presentations, tutorials, workshops, conference platforms, webpage, and official communications. 
While English is indeed regarded as the primary language in scientific communication and one official language makes it conducive to communicate widely, this makes being a native English speaker a privilege.
Non-native English speakers could miss opportunities to attend or actively participate in conferences (e.g. asking questions or participating in discussions),
and conferences may in turn miss innovative contributions.

% Rocío: main idea: make conferences more linguistically inclusive
Providing a welcoming and diverse environment by encouraging the full participation of non-native English speakers is critical (Rule \ref{rule_inclusion}). This may be done at different levels. First, by having spaces in languages other than English, both regarding the technical content of the conference and the social and networking aspects. These exclusive spaces have the benefit of not requiring technical adjustments, but have the problem of not communicating outside their specific public.   
Second, for presentations spoken in English, identify some other key languages for the conference and provide at least English-to-English captions (see Rule \ref{rule_accessibility}), but also translated captions or live interpretation into these key languages. Allow for multilingual Q \& A sessions. Third, include workshops, presentations, and events in languages other than English in the schedule, and provide captions or live interpretation to English. 

When doing this, remember your native English speaking audience to be respectful of accents and English mistakes from non-native English speakers and don't let them be discouraged by having to read captions. 
Promote the attendance to presentations in languages other than English as key components of your schedule and advertise the conference in languages other than English . 


% Rocío: we could eventually cite https://conferenceinference.wordpress.com/2020/11/30/when-language-is-not-a-barrier-a-tale-fr[…]istically-inclusive-conference-toma-pustelnikovaite/

% Not citing ninerBetterWhomLeveling2021 anymore for now.

\section{Express the welcoming spirit in your communication strategy}
% media
\label{rule_communication}

%Rocío: Main idea: Include people in your communication 
Your communication strategy should reflect the spirit of the conference and be inclusive by design. 
As you design the promotion plan, actively reach out and promote the conference to people who have been systematically excluded. 
Include all the work to make the conference a safe and inclusive environment in your communication strategy, 
to emphasize that they* are particularly welcome in this event. This could be done through publishing a diversity statement (e.g., https://user2021.r-project.org/about/diversitystatement/), the Code of Conduct, the accessibility guidelines, and advertising financial support (see Rule \ref{rule_financial}). Try to come up with creative ways to show that everyone is seen, respected, and welcome. \la{For example, useR! 2021 created "Margot, the marmot" a mascot for the conference. Margot is a cartoon with different designs reflecting folks that the conference wanted to reach. In particular, Margot wears different scarfs. The scarfs include colors of flags from different communities in R such as MiR´s, Rainbow R, ..., etc. (Figure 2)."}.
Also, try to advertise the conference in multiple languages (\textbf{Rule \ref{rule_language}}) and multiple platforms (e.g. Twitter, Facebook, LinkedIn, conference website, mailing lists)--
part of your community may use a different platform than the ones adopted by people who regularly attend the conference. 
% allude to the diverse organizing team - to tap into groups, contexts and make choices on platforms and ways to communicate
With proactive inclusive communication strategies, potential attendees will be assured that this is their space and community too.

%Rocío: Main idea: Use inclusive language
Inclusive language--language free from words, phrases or tones that reflect prejudiced or discriminatory views of particular people or groups--should be used in all communications \cite{hallDesigningDiversityInclusion2019}.
Make the effort to teach yourself the vocabulary and the best ways to communicate to account for every culture and situation; do not expect minoritized people to teach you--it's not their role--and accept feedback without being offended.



% Rocío: could add some ideas from here 
% https://www.science.org.au/files/userfiles/support/emcr/documents/one-page-summary-emcr-improving-diversity-web.pdf

\section{Allocate adequate financial resources to support your conference goals}
\label{rule_financial}

% Rocío: main idea: Budgets are limited and need to choose priorities.
Conference budgets are limited and rely mostly on sponsorship and attendance fees. 
While some expenses might be more or less fixed or known, (e.g., venue, internet platforms, catering) % Rocío: can we give examples?
% Dorothea: this depends so much on the conference type. I wanted to write "venue"but this is not true of course, so I don't know what to add. %ast i added a couple more, i meant the most "obvious" budget items  in contrast to the less obvious
allocation of resources has to be intentional to support the goals of inclusion of the conference. 
It is important to estimate the costs for these practices and define your priorities in advance (e.g. paying the organizing team, Code of Conduct training, captioning).
Additional support for attendees should also be considered in the budget: child care support, transportation fees, visa-related support (if in-person), internet connection services (for the virtual component). 
Consider that an online conference might reduce organization costs (e.g. no rental costs for a physical venue), allowing to redirect the money towards other inclusive priorities. 
When asking for sponsorship, it might be easier to justify supporting concrete actions towards inclusion, than making generic demands for funding.
%Dorothea: I slightly disagree with that one. Maybe just leave the first part out?

% Rocío: main idea: help take the burden out of participants for more inclusive participation
Registration costs are one of the largest barriers for conference attendance, and, if we are aiming for inclusiveness and representation, the socioeconomic context of participants, their country of origin, and their career status should be taken into account when determining the registration rates \cite{sarabipourChangingScientificMeetings2021, andalibPostdocQueueLabour2018, kaplanPostdocNot2012}
(see \cite{canelon2021cost} for an example of conversion rates based on country of origin and career status). 
In general, people should have the option to locate themselves in a category they consider affordable, even with the possibility of a `pay what you can' approach. 
Resources permitting, you can aim to have a conference with no registration costs. This is especially true for online events, but bear in mind that free events have a lower attendance rate than non-free events \cite{eventbrite_ultimate_2017}. 
%[here anything that might be general support and may allow to help all attendants, fee waivers, ]
%Offering fee waivers to attendees is also a good option, but even then, other costs can be prohibitive. 
Scholarships or grants to attend the conference are an additional way to boost participation of people from marginalized groups, by offering support for travel and lodging expenses, in in-person settings.
Design the diversity-related grants to be explicit about the groups you want to support, to avoid self-selection (the fact that some people refrain from applying because they think they don't stand a chance, while privileged people apply even when they do not need the support). 
Conferences usually ask for cover letters or applications to assign these grants, and this can be a time-consuming, emotionally-demanding task. Simplify as much as you can the process of asking for financial support. People who do not have the time (e.g., because they have care-taking roles) might decide not to apply if the process is overtly complicated. \la{Bear in mind that transferring money internationally is a cumbersome administrative process that can significantly augment the burden of already marginalized groups. Whenever possible, facilitate ways around money transfer (e.g., book flights, hotel reservations, waive conference fees).}
%Dorothea: I understand what we want to say, but there is a slight contradiction between "reduce the prestige so that less people apply and we can help everyone and reduce the barriers so that more people apply"


%For online conferences, granting fee waivers or support for hidden costs like internet connection is easier and cheaper. 

% Rocío: Let's keep this in mind for the 10th rule or conclusion
% While some of these items are being implemented recently, others can encounter a certain degree of resistance, such as paying the organizing team. Don't be afraid to innovate and resist the institutional inertia. "We have never payed for this, and this has always been like that" will not take the conference towards structural change.
% %ast should we talk about the pandemic? something like: 
% With the COVID-19 pandemic and the shift to online (or hybrid) events, some expenses disappeared, leaving space for rearrangements in this sense.
% Depending on registration fees to support large budget items is not at all desirable, instead sponsors can be asked to support specific items in this list.


\section{Diversity and inclusion are processes}
Make the conference part of a long-term process for inclusion
\label{rule_process}
%continuity, accountability, learning and being open to share the info, the teahcings, transmit what you learnt and share the list of people, expand the networks, tell others what was difficult - 
% Rocío: main idea: we're not changing everything at the beginning, the conference will not be perfectly inclusive
Do not expect the conference to be perfect. 
Resources, both in time and money, are always limited. 
Most importantly, there are systemic discrimination issues at higher levels (e.g. society, academia) that one conference cannot change. 
\la{Due to these systemic problems, even the minimum changes to how conferences have been historically organized will find resistance}, but your courage to make changes will be a step towards a more diverse and inclusive community, and can have a huge impact in the lives and careers of often excluded and minoritized people.

% Rocío: main idea: be part of the process
Diversity and inclusion will be the norm in the long-term, if we see conferences as part of the required structural change. 
After the conference is over, assess whether equity and inclusion goals were met during the meeting. 
This can be done internally within the organizing team and taking into account the attendees' point of view. 
Identify the things that worked towards your inclusion goals, and try to understand what went wrong and how people were excluded.
Importantly, make sure to share the information with future organizers so that they can use it as a starting point and keep working to improve inclusion in the next edition of the conference. 

% For many people working with diversity, one typical aim is institutionalizing diversity (Ahmed 2012). However, our aim is also a symptom that diversity and equity are not part of institutions, academic societies, and conferences. This rule is about the process of making diversity and inclusion possible and a complement to all the other rules. We believe we need to be accountable and we need to aim for permanence. Institutionality comes from a set of norms, values, and priorities and these determine who is allowed and how (Ahmed 2012). Our long-term goal is to set new norms, values, and priorities based on diversity. Organizing a diverse conference is part of the process of change. However, we cannot do everything at one conference or a few of them. It is important to have a bold vision and clear targets. Keep in mind that ideal actions are often not possible and resources are limited (both time of the organizing team and money. 

% If from the beginning we set diversity goals and actions for the conference, we can use that to be transparent and, later, accountable. Assess whether equity and inclusion goals were met during the meeting. This can be done internally within the organizing team and often also through the conference’s participants. Plan, balance and keep working to improve inclusion in the next edition of the conference. Try to propose permanent data collection to assess whether diversity and inclusion are actually being seen at the conference. We know that diversity and equity are not institutionalized, in fact, the system works to stop those who are changing the system (Ahmed 2019). We will find resistance, but diversity is a school. We also learn more about how people are being excluded, we learn that there are actions we can take to include more, and then, we can set higher goals on how to make inclusion possible.

\section*{Concluding remarks}

This article suggests rules to pivot traditional conferences towards diversity and inclusion, and strive to build more inclusive and welcoming communities. 
This set of ten rules stated here can be adapted depending on the conference format and settings.
We organized useR! during a global pandemic, and as a team, this was a challenging journey. 
We engaged in most of the practices mentioned here and learned others along the way. 

Organizing a conference and implementing inclusive practices are both learning experiences;
do not expect to achieve everything, set priorities and remember rule 10: diversity and inclusion are processes. 
If you are part of a stable meetings committee, you could encourage organizers to follow these rules. 
If more conferences and domains apply them, the process will get more streamlined, straightforward, and mainstream to adapt with minimal overhead.
And you will make a change towards healthier, stronger, and more inclusive communities.


\section*{Acknowledgments}
The authors of this piece would like to thank every single member of the organizing team of useR! 2021 [ \url{https://user2021.r-project.org/about/global-team/}] for their valuable contribution to an inclusive conference experience, and the R Foundation for trusting us with the organization of useR! 2021 and supporting us through the process. 


% \nolinenumbers

% % Either type in your references using
% % \begin{thebibliography}{}
% % \bibitem{}
% % Text
% % \end{thebibliography}
% %
% % or
% %
% % Compile your BiBTeX database using our plos2015.bst
\bibliography{community-science}
% % style file and paste the contents of your .bbl file
% % here. See http://journals.plos.org/plosone/s/latex for 
% % step-by-step instructions.
% % % 
% \begin{thebibliography}{10}

% \bibitem{bib1}
% Conant GC, Wolfe KH.
% \newblock {{T}urning a hobby into a job: how duplicated genes find new
%   functions}.
% \newblock Nat Rev Genet. 2008 Dec;9(12):938--950.

% \bibitem{bib2}
% Ohno S.
% \newblock Evolution by gene duplication.
% \newblock London: George Alien \& Unwin Ltd. Berlin, Heidelberg and New York:
%   Springer-Verlag.; 1970.

% \bibitem{bib3}
% Magwire MM, Bayer F, Webster CL, Cao C, Jiggins FM.
% \newblock {{S}uccessive increases in the resistance of {D}rosophila to viral
%   infection through a transposon insertion followed by a {D}uplication}.
% \newblock PLoS Genet. 2011 Oct;7(10):e1002337.

% \end{thebibliography}



\end{document}


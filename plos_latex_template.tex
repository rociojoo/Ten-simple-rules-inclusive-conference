% Template for PLoS
% Version 3.5 March 2018
%
% % % % % % % % % % % % % % % % % % % % % %
%
% -- IMPORTANT NOTE
%
% This template contains comments intended 
% to minimize problems and delays during our production 
% process. Please follow the template instructions
% whenever possible.
%
% % % % % % % % % % % % % % % % % % % % % % % 
%
% Once your paper is accepted for publication, 
% PLEASE REMOVE ALL TRACKED CHANGES in this file 
% and leave only the final text of your manuscript. 
% PLOS recommends the use of latexdiff to track changes during review, as this will help to maintain a clean tex file.
% Visit https://www.ctan.org/pkg/latexdiff?lang=en for info or contact us at latex@plos.org.
%
%
% There are no restrictions on package use within the LaTeX files except that 
% no packages listed in the template may be deleted.
%
% Please do not include colors or graphics in the text.
%
% The manuscript LaTeX source should be contained within a single file (do not use \input, \externaldocument, or similar commands).
%
% % % % % % % % % % % % % % % % % % % % % % %
%
% -- FIGURES AND TABLES
%
% Please include tables/figure captions directly after the paragraph where they are first cited in the text.
%
% DO NOT INCLUDE GRAPHICS IN YOUR MANUSCRIPT
% - Figures should be uploaded separately from your manuscript file. 
% - Figures generated using LaTeX should be extracted and removed from the PDF before submission. 
% - Figures containing multiple panels/subfigures must be combined into one image file before submission.
% For figure citations, please use "Fig" instead of "Figure".
% See http://journals.plos.org/plosone/s/figures for PLOS figure guidelines.
%
% Tables should be cell-based and may not contain:
% - spacing/line breaks within cells to alter layout or alignment
% - do not nest tabular environments (no tabular environments within tabular environments)
% - no graphics or colored text (cell background color/shading OK)
% See http://journals.plos.org/plosone/s/tables for table guidelines.
%
% For tables that exceed the width of the text column, use the adjustwidth environment as illustrated in the example table in text below.
%
% % % % % % % % % % % % % % % % % % % % % % % %
%
% -- EQUATIONS, MATH SYMBOLS, SUBSCRIPTS, AND SUPERSCRIPTS
%
% IMPORTANT
% Below are a few tips to help format your equations and other special characters according to our specifications. For more tips to help reduce the possibility of formatting errors during conversion, please see our LaTeX guidelines at http://journals.plos.org/plosone/s/latex
%
% For inline equations, please be sure to include all portions of an equation in the math environment.  For example, x$^2$ is incorrect; this should be formatted as $x^2$ (or $\mathrm{x}^2$ if the romanized font is desired).
%
% Do not include text that is not math in the math environment. For example, CO2 should be written as CO\textsubscript{2} instead of CO$_2$.
%
% Please add line breaks to long display equations when possible in order to fit size of the column. 
%
% For inline equations, please do not include punctuation (commas, etc) within the math environment unless this is part of the equation.
%
% When adding superscript or subscripts outside of brackets/braces, please group using {}.  For example, change "[U(D,E,\gamma)]^2" to "{[U(D,E,\gamma)]}^2". 
%
% Do not use \cal for caligraphic font.  Instead, use \mathcal{}
%
% % % % % % % % % % % % % % % % % % % % % % % % 
%
% Please contact latex@plos.org with any questions.
%
% % % % % % % % % % % % % % % % % % % % % % % %

\documentclass[10pt,letterpaper]{article}
\usepackage[top=0.85in,left=2.75in,footskip=0.75in]{geometry}

% amsmath and amssymb packages, useful for mathematical formulas and symbols
\usepackage{amsmath,amssymb}

% Use adjustwidth environment to exceed column width (see example table in text)
\usepackage{changepage}

% Use Unicode characters when possible
\usepackage[utf8x]{inputenc}

% textcomp package and marvosym package for additional characters
\usepackage{textcomp,marvosym}

% cite package, to clean up citations in the main text. Do not remove.
\usepackage{cite}

% Use nameref to cite supporting information files (see Supporting Information section for more info)
\usepackage{nameref,hyperref}

% line numbers
\usepackage[right]{lineno}

% ligatures disabled
\usepackage{microtype}
\DisableLigatures[f]{encoding = *, family = * }

% color can be used to apply background shading to table cells only
\usepackage[table]{xcolor}

% array package and thick rules for tables
\usepackage{array}

%strikethrough
\usepackage{soul}

% create "+" rule type for thick vertical lines
\newcolumntype{+}{!{\vrule width 2pt}}

% create \thickcline for thick horizontal lines of variable length
\newlength\savedwidth
\newcommand\thickcline[1]{%
  \noalign{\global\savedwidth\arrayrulewidth\global\arrayrulewidth 2pt}%
  \cline{#1}%
  \noalign{\vskip\arrayrulewidth}%
  \noalign{\global\arrayrulewidth\savedwidth}%
}

% \thickhline command for thick horizontal lines that span the table
\newcommand\thickhline{\noalign{\global\savedwidth\arrayrulewidth\global\arrayrulewidth 2pt}%
\hline
\noalign{\global\arrayrulewidth\savedwidth}}


% Remove comment for double spacing
%\usepackage{setspace} 
%\doublespacing

% Text layout
\raggedright
\setlength{\parindent}{0.5cm}
\textwidth 5.25in 
\textheight 8.75in

% Bold the 'Figure #' in the caption and separate it from the title/caption with a period
% Captions will be left justified
\usepackage[aboveskip=1pt,labelfont=bf,labelsep=period,justification=raggedright,singlelinecheck=off]{caption}
\renewcommand{\figurename}{Fig}

% Use the PLoS provided BiBTeX style
\bibliographystyle{plos2015}

% Remove brackets from numbering in List of References
\makeatletter
\renewcommand{\@biblabel}[1]{\quad#1.}
\makeatother



% Header and Footer with logo
\usepackage{lastpage,fancyhdr,graphicx}
\usepackage{epstopdf}
%\pagestyle{myheadings}
\pagestyle{fancy}
\fancyhf{}
%\setlength{\headheight}{27.023pt}
%\lhead{\includegraphics[width=2.0in]{PLOS-submission.eps}}
\rfoot{\thepage/\pageref{LastPage}}
\renewcommand{\headrulewidth}{0pt}
\renewcommand{\footrule}{\hrule height 2pt \vspace{2mm}}
\fancyheadoffset[L]{2.25in}
\fancyfootoffset[L]{2.25in}
\lfoot{\today}

%% Include all macros below

\newcommand{\lorem}{{\bf LOREM}}
\newcommand{\ipsum}{{\bf IPSUM}}

%% END MACROS SECTION


\begin{document}
\vspace*{0.2in}

% Title must be 250 characters or less.
\begin{flushleft}
{\Large
\textbf\newline{Ten simple rules towards an inclusive conference} % Please use "sentence case" for title and headings (capitalize only the first word in a title (or heading), the first word in a subtitle (or subheading), and any proper nouns).
}
\newline
% Insert author names, affiliations and corresponding author email (do not include titles, positions, or degrees).
\\
Name1 Surname\textsuperscript{1,2\Yinyang},
Name2 Surname\textsuperscript{2\Yinyang},
Name3 Surname\textsuperscript{2,3\textcurrency},
Name4 Surname\textsuperscript{2},
Name5 Surname\textsuperscript{2\ddag},
Name6 Surname\textsuperscript{2\ddag},
Name7 Surname\textsuperscript{1,2,3*},
with the Lorem Ipsum Consortium\textsuperscript{\textpilcrow}
\\
\bigskip
\textbf{1} Affiliation Dept/Program/Center, Institution Name, City, State, Country
\\
\textbf{2} Affiliation Dept/Program/Center, Institution Name, City, State, Country
\\
\textbf{3} Affiliation Dept/Program/Center, Institution Name, City, State, Country
\\
\bigskip

% Insert additional author notes using the symbols described below. Insert symbol callouts after author names as necessary.
% 
% Remove or comment out the author notes below if they aren't used.
%
% Primary Equal Contribution Note
\Yinyang These authors contributed equally to this work.

% Additional Equal Contribution Note
% Also use this double-dagger symbol for special authorship notes, such as senior authorship.
\ddag These authors also contributed equally to this work.

% Current address notes
\textcurrency Current Address: Dept/Program/Center, Institution Name, City, State, Country % change symbol to "\textcurrency a" if more than one current address note
% \textcurrency b Insert second current address 
% \textcurrency c Insert third current address

% Deceased author note
\dag Deceased

% Group/Consortium Author Note
\textpilcrow Membership list can be found in the Acknowledgments section.

% Use the asterisk to denote corresponding authorship and provide email address in note below.
* correspondingauthor@institute.edu

\end{flushleft}
% Please keep the abstract below 300 words
\section*{Abstract (optional from what I've seen)}

In July 2021, the authors of this article, along with a larger group of people, organized the annual user conference of the R Project for Statistical Computing. This instance of the useR! conference was particularly well received as a virtual and explicitly global event, reaching users and developers of the R language from more than 120 countries. useR! conferences are non-profit events, organized by volunteers from the R community and supported by the R Foundation. The conference attracts a broad range of participants from academia, industry, government and the non-profit sector. For 2021, we aimed to build a high-quality conference in a kind, inclusive, accessible, and welcoming environment for everyone. In this article, we share a few critical lessons learned in the process. We streamline our most important learnings in 10 simple rules to host an inclusive conference. These rules apply equally to academic, industry, or mixed conferences; the rules are inspired by a global experience but also apply at the regional or local level.


% % Please keep the Author Summary between 150 and 200 words

\linenumbers

\section*{Introduction}

Conferences are spaces to meet and reconnect with members from a specific community, learn about advances in the field and share our recent contributions. The larger the conference, the larger the opportunities to network and learn from your cohort. However, conferences can become an exclusive space for privileged groups (e.g., white, male, from a rich country, English-native speaker, with no physical disabilities)  \cite{arend_disparity_2019, timperley_he_2020, gewin_what_2019, brown_ableism_2018}.

Calls to improve diversity in many spaces, including conferences, are becoming more frequent as communities and scientific societies are getting called out for over-representing specific groups. While there has been some action to address systematic inequalities, there is still a lot of room for improvement.

This article suggests rules to pivot traditional conferences towards inclusiveness and diversity and to welcome minoritized groups. These tips stem from the authors' experience of organizing a virtual and global conference for users and developers of the R programming language, useR! 2021. We embraced the challenge of organizing a high-quality conference in the context of the COVID-19 pandemic using the chance of a virtual conference environment to create a kind, truly inclusive, accessible, and welcoming experience for everyone. Most of the authors also have experience of  organizing other regional and national academic conferences and events in communities of practice such as R-Ladies and The Carpentries. Here, we share the lessons learned within the past year of organizing this global useR! conference, summarized as 10 simple rules towards an inclusive conference.

\section{Rule X: Define what diversity and inclusion mean for your conference}

The first step towards a diverse and inclusive conference is to recognize that people are diverse, that some groups face discrimination and might be underrepresented in your community and event \cite{timperley_he_2020}. 

Diversity encompasses multiple dimensions: age, physical ability, career stage, gender, gender identity,  geographic origin, language, neurodiversity, race, religion, sexual orientation, and socioeconomic background, to name a few. There are implicit hierarchies along these axes, and some statuses (e.g., cisgender, white, male, from the US or Europe) hold the privilege of being the default settings for which all systems are consciously and unconsciously built. Thinking about diversity and inclusion is thinking about counteracting the structures that hold these hierarchies in place.

As a starting point, identify which groups are usually unsupported or discriminated against in your community or event, and determine the contributing factors. Next, think about set measurable indicators of success when organizing a diverse conference. The numbers resulting from these indicators are not the primary goal. Instead, this vision should guide and help hold the organizing team accountable towards these diverse and inclusive goals.

Publishing a diversity statement (e.g., \url{https://user2021.r-project.org/about/diversity_statement/}) and expressing this welcoming spirit in your communication strategy (social networks, website, brand, and visual identity) and in every decision you make, will let people know that they are seen, respected, and welcome; that this is their space and community too. However, it is important to not stop with a statement alone, but to identify specific goals and tasks that ensure compliance with these larger goals, for example, identifying accessible platforms, developing a detailed code of conduct, and preparing accessibility guidelines (see other Rules below).


\section{Rule X: Have a strong online component of the conference} 

In-person interaction is priceless; however, it is more expensive for some, even unattainable for others. This inaccessibility is particularly true for global conferences that usually take place in high-income countries, making it financially demanding for international participants and often impossible to attend due to immigration requirements \cite{arend_disparity_2019, gewin_what_2019}. Online conferences are more inclusive: they do not need a visa or a big budget, and are more accessible to people who may be unable to travel because of health issues or family responsibilities. Furthermore, an online format is more environmentally friendly since it eliminates travel-related emissions \cite{sarabipour_evaluating_2020, niner_better_2021}; in a study case, conference attendance accounted for $35\%$ of a researchers footprint .

Alternatively, a conference could have a hybrid format with an in-person and online component. This dual format could allow a group of people to interact face-to-face while providing many others the opportunity to participate remotely. The challenge in this kind of setting would be to make the online component equally relevant as the in-person component and not just a consolation prize to the less privileged in the community \cite{niner_better_2021}. 


\section{Rule X: Have an inclusive and diverse organizing team}
A genuinely inclusive conference can only be organized by an inclusive and diverse organizing team. Build a team with people from different regions, genders, socioeconomic statuses, and other aspects of diversity. Particular attention should be paid to the usually underrepresented groups (see \textbf{Rule 1} discussing the dimensions of diversity lacking in your community and event). To ensure a deeper understanding of the challenges in different diverse groups, it is essential to create a representative working group that functions as a snapshot of the community at large. Even if creating and maintaining such an interdisciplinary team is challenging, the positive outcomes far outweigh any minor inconveniences, and the team will grow, learn, and embrace more inclusive practices. For instance, if the organizing team is global and remote, time zones can be hard to manage, with meetings scheduled at odd hours. The team may also have to review carefully and teach themselves the correct vocabulary for internal and external communication and identify the best ways to account for every culture and situation. 
An inclusive team will let everyone know if the organization is failing in any aspect of inclusiveness or accessibility, and help improve it \cite{hong_groups_2004}.

%rj: Yani's refs to have just in case:
%https://sites.lsa.umich.edu/scottepage/wp-content/uploads/sites/344/2015/11/pnas.pdf; https://www.pnas.org/content/early/2014/11/13/1407301111; https://www.mckinsey.com/business-functions/organization/our-insights/why-diversity-matters

\section{Rule X: Search proactively the best people for each task}

While this rule may seem intuitive, one would be surprised by how often conference organizers tend to do everything themselves instead of reaching out to the right people for the right task -- likely because the latter involves careful planning and communication well ahead of time or because they naively overestimate their expertise in areas they lack practical experience in. For example, a cisgender or abled person estimating what might work best for transgender people or people with sensory disabilities. 
Therefore, to build your team of organizers for the conference, you need to actively seek out and recruit the best people for each task. The intention to put together a diverse team alone is not enough. One needs to look for, reach out to, invite, encourage, and onboard these people until there is ample representation across the diversity spectrum and dimensions [\textbf{Fig. 1}]. To achieve this, it will be necessary to go beyond one's limited networks when inviting people to the organizing team, keynote speakers, program committee, session chairs, reviewers, and other roles. Ensuring diversity in each of these teams and panels is a deliberate process -- if there is not much diversity in the first set of names, it needs to be revised to include people from minoritized groups. If these groups aren't showing up in the A-list, it is seldom because of their technical acumen -- it is because we, as organizers, are not used to turning every work and looking in unfamiliar places to bring in the talent.  Many of our biased lists are products of the existing systems that have always privileged some groups of people \cite{dwyer_notice_2021, sarabipour_evaluating_2020}, thus encouraging us to look further to find great people that are not routinely in the spotlight. We need to go beyond our narrow and often limited networks. So, as long as we start with an inclusive and diverse organizing team (\textbf{Rule 2}), we can aim at finding the right people for various aspects of the conference organization. The regional and local communities are also good sources to tap into. For example, for useR! 2021, groups like AfricaR, R-Ladies, MiR, Forwards, and LatinR were fundamental to reach people for the organizing team, potential attendees, and sponsors.

Disabled people often repeat "Nothing about us without us". This means the actual life experiences of people from URM are not replaceable with good intentions from people outside these groups. The same holds for other dimensions of diversity. No space will be truly inclusive and welcoming to URM without people from these groups bringing their experience to bear. This means people in privileged groups may need to step down from positions of power. Of course, no-one has the same experiences and experiences the same oppressions in life. %When building a diverse team, bear in mind that 

\section{Rule X: Set registration rates commensurate with the registrant's cost of living}

Conferences, even virtual ones, should have registration fees for two primary reasons. The first one is that preparing the conference involves a lot of concerted effort and costs money (e.g., commercial registration tools, captioning, or conference venue, if in-person) and this translates to the participation fee. The second reason has to do with psychology -- people value more the things they pay for, and there is a lower attendance rate for free events with no registration costs \cite{eventbrite_ultimate_2017}. On the other hand, if we are aiming for inclusiveness and representation, the socioeconomic context of participants should be taken into account when determining the registration rates. Usually, there is a higher fee for people from the industry than for academia ($\sim$ proportionate to income). We should also consider a lower fee for non-profit organizations, government employees, or freelancers, and use conversion factors (for country of residence using data from International Comparison Program report of the World Bank \cite{arend_disparity_2019}). It is important to include discounts for students as well as for postdocs (early career researchers) to encourage their participation \cite{sarabipour_evaluating_2020, andalib_postdoc_2018, kaplan_postdoc_2012}. Since postdoc (or trainee) statuses are not always well-defined in academia and can vary for each country, their payment category should be explicitly defined. Disabilities may also increase the overall cost of conferences, especially in-person \cite{de_picker_rethinking_2020, irish_increasing_2020}; discounts would help reduce the financial burden for these factions too.

Representation is an important aspect of diversity and inclusivity: seeing people like you, from your country, that speak your language is one of the best way to feel you belong, this is why the rates can not only be different for registration, but also, for sponsors, this allows us to have institutions and companies from all over the world being part of the conference which also means they will communicate about our event in their local communities.


\section{Rule X: Abandon diversity scholarships}

%rationale: 1. they might select for the already privileged or people who dont need those to be nice on the cvs 2. they select for outgoing extroverted privileged ("brave") people 3. they are time and emotionally consuming which selects out some people 4. they single out people. even if you controlled for 1 to 3 you would "badge" people - you are creating a minority while trying to avoid it.
Offering scholarships or grants to attend the conference may be counterproductive, as they can be seen as more of a competition and attract people that might not need the money (e.g., when their PI could pay for it) but could apply for it to enrich their CV while filtering out people who do not feel entitled to earn them. If the goal is to help the people who really need it, the language needs to make that clear without shrouding it with fancy-sounding phrases. Ultimately, it is a fee waiver or a discount for the people who need additional financial support. Also, the process for applicants should be simplified. People with limited resources already have a hard time applying for loans, grants, and scholarships. We should not make the attendees' lives more challenging by asking for long essays to convince us that they deserve our support. \st{Many people that cannot afford international conferences do not even try to apply for waivers/scholarships because they are discouraged by the requirements, the process, and partly because they believe it would be hard for them to win financial/travel aid.} For this stage of the process, a certain degree of trust goes a long way -- when they say they need the support, it is best to give them the waiver rather than second guess their eligibility. Also, the participant metadata (e.g., demographics, gender, career stage) is already a strong indicator of their potential need for aid.
If the conference resources allow it, the organizers could even take further steps to offer financial support for activities that help them have the time and resources to be at the conference: child care support, transportation fees (if in-person), or internet connection services (if virtual). 

\section{Rule X: Make the conference accessible to all, from start to end}

Conferences are often inaccessible to people with hearing, visual, and other disabilities \cite{price_access_2009}. People may also find conferences inaccessible due to their caring responsibilities, dietary restrictions, religious practices, or LGBTQ+ status. Each of these groups should be considered at every stage of the planning process. Welcoming representatives from these groups into the organizing team would allow them to take part in the decisions from the beginning and spot right away the inaccessible practices that need to improve (see \textbf{Rule 2}). If the conference is in person, the venue must be accessible for people with mobility limitations. Accommodations such as a quiet space for neurodivergent people and a space for breastfeeding should also be provided. Images used in communications about the conference should be accompanied by alternative text, while videos should have both captions and sound. Platforms for conference registration and abstract submission, websites and chat platforms should be screen-reader friendly and keyboard accessible, with low technology requirements (hardware, software, and internet connection). These aspects should be tested well in advance of going live. Captioning for presentations should be available in more than one language if possible. If the conference is in-person, presenters should always speak into a microphone to make it easier for the hard of hearing and for captioners or interpreters to listen to them. Accessibility guidelines for slides and presentations should be provided to the presenters, and their use should be encouraged (see \url{https://user2021.r-project.org/participation/accessibility/} for example). Among other things, the speakers should provide raw and accessible material to their talks before the conference, e.g., R Markdown, HTML, or \TeX{} files; if they prerecord their presentation, they should include their video and ensure that their face is visible so that deaf and hard-of-hearing people can read their lips if needed. Adding captions to recorded videos is relatively easy these days and is helpful to deaf and hard-of-hearing people, non-native speakers, and everyone in general.  Accessibility awards at the end of the conference are an excellent way to acknowledge the speakers who were mindful of inclusiveness when preparing and delivering their talks. 

Accessibility practices are not afterthoughts that can be dealt with at the last minute. They require time and early decision-making \cite{irish_increasing_2020}. Conversely, inaccessible decisions are hard to course-correct, e.g., when finding out too late that a venue is inaccessible for wheelchairs in an in-person conference. 
Bear in mind that most accessibility practices are beneficial to everyone. Captions, for instance, are used by non-native speakers and people that access the conference in loud environments. The prior availability of material benefits people with low bandwidth and non-native speakers as well.


\section{Rule X: Take deliberate efforts to promote the conference among underrepresented groups}

If some minoritized groups are underrepresented in the conference, they may not feel like they belong. 
If part of the community has been historically discriminated against, one should emphasize that they are particularly welcome in this event and that the organization will make it a safe and inclusive environment. 
An effort must be made to promote the conference as welcoming and supportive for those without the necessary 'credentials' or resources. For the latter, fee waivers, financial support, and ease of application/registration should be advertised. People who are used to applying and getting rejected for scholarships and waivers may find it relieving that the process will be supportive rather than discriminatory against underrepresented groups. Providing financial and other support (e.g., childcare, flexible hours) will help minoritized groups partake more willingly in the organization and the actual conference.
%ast likewise, some key people in your organizing team might not be able to collaborate without remuneration. Resources permitting, it may be a good idea to secure funds to offer stipends to them.

\section{Rule X: Don't let language restrict high-quality participation}

In international conferences, English is often the official language. Submissions, presentations, tutorials, and workshops are in English. The platforms, the webpage, and official communications are also in English. While English is the primary language in scientific communication and one official language makes it conducive to communicate widely, opening up the conference to other languages could make it less intimidating to people who are not fluent in English \cite{niner_better_2021}. Excluding them may potentially lead to missing innovative contributions due to a language barrier. Advertising the conference in several languages and considering having non-English workshops and presentations (with or without captions in English) could help overcome this barrier. For instance, hosting one international day/session per conference might be a great place to start!

\section{Rule X: Have a code of conduct and a team to reinforce it}

To ensure an inclusive, friendly, and safe space in the conference, the organizers need to adopt a code of conduct and set up a team to enforce it \cite{favaro_your_2016}. The code of conduct is a document meant to keep the community safe and should state clearly the unacceptable behaviors, the spaces of the conference in which it applies, the consequences for engaging in unacceptable behavior, and the way to report violations \cite{aurora_how_2018}.%The people who need the protection of a code of conduct are usually those with less power or privilege, as more powerful or privileged people are often already protected from most harm.' 
The code of conduct should be displayed prominently in several spaces of the conference to deter people from unacceptable behavior.

The code of conduct team should receive training on how to receive reports, respond to incidents, and communicate their responses. A diverse code of conduct team will be more understanding of intersectionality issues in discrimination and harassment practices. 



\section*{Concluding remarks}

The ten rules stated here can be adapted depending on the conference format and settings. From our own experience, we are aware that enforcing these rules requires a tremendous amount of work that is generally not compensated financially or otherwise. However, the rewards from these suggested best practices will provide a healthier, stronger, and more inclusive community, empowering historically marginalized people and learning from them, thereby increasing the overall quality of the conference. Since these are early days and accessibility is not as widespread as it should be, the process might seem exacting. Hopefully, if more conferences and domains employ these techniques, the process will get more streamlined, straightforward, and mainstream to adapt with minimal overhead. After all, the strength of any community lies in its ability to be adaptive, inclusive, and accessible.


\section*{Acknowledgments}
The authors of this piece would like to thank every single member of the organizing team of useR! 2021 [ \url{https://user2021.r-project.org/about/global-team/}] for their valuable contribution to an inclusive conference experience, and the R Foundation for charging us with the organization of useR! 2021 and supporting us through the process. 

% Cras egestas velit mauris, eu mollis turpis pellentesque sit amet. Interdum et malesuada fames ac ante ipsum primis in faucibus. Nam id pretium nisi. Sed ac quam id nisi malesuada congue. Sed interdum aliquet augue, at pellentesque quam rhoncus vitae.

% \nolinenumbers

% % Either type in your references using
% % \begin{thebibliography}{}
% % \bibitem{}
% % Text
% % \end{thebibliography}
% %
% % or
% %
% % Compile your BiBTeX database using our plos2015.bst
\bibliography{references}
% % style file and paste the contents of your .bbl file
% % here. See http://journals.plos.org/plosone/s/latex for 
% % step-by-step instructions.
% % % 
% \begin{thebibliography}{10}

% \bibitem{bib1}
% Conant GC, Wolfe KH.
% \newblock {{T}urning a hobby into a job: how duplicated genes find new
%   functions}.
% \newblock Nat Rev Genet. 2008 Dec;9(12):938--950.

% \bibitem{bib2}
% Ohno S.
% \newblock Evolution by gene duplication.
% \newblock London: George Alien \& Unwin Ltd. Berlin, Heidelberg and New York:
%   Springer-Verlag.; 1970.

% \bibitem{bib3}
% Magwire MM, Bayer F, Webster CL, Cao C, Jiggins FM.
% \newblock {{S}uccessive increases in the resistance of {D}rosophila to viral
%   infection through a transposon insertion followed by a {D}uplication}.
% \newblock PLoS Genet. 2011 Oct;7(10):e1002337.

% \end{thebibliography}



\end{document}

